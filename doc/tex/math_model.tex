
\section{Model matematyczny}

Model matematyczny.

\subsection{Notacja}

Notacja dotycząca modelu.

\begin{table}[!ht]
\centering
\begin{tabular}{|c|l|}
\hline
$T$ & minimalizowany czas zakończenia wszystkich dzialań \\ \hline
$T_{x}$ & najpóźniejszy czas zakończenia działań na procesorze x \\ \hline
$A_{x}$ & czas potrzebny na obliczenie jednej porcji danych na procesorze x \\ \hline
\end{tabular}
\caption{Notacja wykorzystywana w raporcie}
\label{tab:notacja}
\end{table}

\subsection{Wzory wspólne dla wszystkich modeli}

We wszystkich modelach minimalizujemy najdłuższy czas obliczeń

\begin{equation} \label{eq:min}
\begin{array}{l}
min T \\
T \geq T_{1} \\
T \geq T_{2} \\
T \geq T_{3} \\
T \geq T_{4}
\end{array}
\end{equation}

Istotne są założenia odnośnie danych

\begin{equation} \label{eq:data}
\begin{array}{l}
d_{1} + d_{2} + d_{3} + d_{4} = V \\
d_{211} + d_{212} = d_{2} \\
d_{321} + d_{322} = d_{3} \\
d_{43} + d_{42} = d_{4} \\
d_{421} + d_{422} = d_{42} \\
d_{431} + d_{432} = d_{43}
\end{array}
\end{equation}

Dla wszystkich modeli można zastosować te same wzory na czas komunikacji na poszczególnych łączach.

\begin{equation} \label{eq:tk}
\begin{array}{l}
t_{k12} = S_{12}x_{12} + (V - d_{1})C_{12} \\
t_{k23} = S_{23}x_{23} + C_{23}d_{3} + C_{23}d_{43} \\
t_{k24} = S_{24}x_{24} + C_{24}d_{42} \\
t_{k34} = S_{34}x_{34} + C_{34}d_{43}
\end{array}
\end{equation}

W powyższych wzorach wykorzystywane są zmienne binarne. Są one tak dobrane, że jeśli między procesorami $a$ i $b$ występuje komunikacja,
to $t_{kab} = 1$, w przeciwnym wypadku solver może przypisać im wartość $0$ w celu optymalizacji rozwiązania.

\begin{equation} \label{eq:x}
\begin{array}{l}
x_{12} < 1 \\
Vx_{12} > d_{2} + d_{3} + d_{4} \\
x_{23} < 1 \\
Vx_{23} > d_{3} + d_{43} \\
x_{24} < 1 \\
Vx_{24} > d_{42} \\
x_{34} < 1 \\
V_{x34} > d_{43}
\end{array}
\end{equation}

Istotne jest, żeby zmienne dotyczące ilości danych oraz zmienne binarna były całkowite.

\subsection{Modele sekwencyjne}

\subsubsection{Pierwszy model sekwencyjny}

\begin{equation} \label{eq:sek1}
\begin{array}{l}
T_{1} \geq t_{k12} + d_{1}A_{2} \\
T_{2} \geq t_{k12} + t_{k24} + t_{k23} + d_{2}A_{2} \\
T_{3} \geq t_{k12} + t_{k24} + max(t_{k23} + d_{321}A_{3}, d_{421}A_{4}) + t_{k34} + d_{322}A_{3} \\
T_{4} \geq t_{k12} + t_{k24} + max(t_{k23} + d_{321}A_{3}, d_{421}A_{4}) + t_{k34} + (d_{422} + d_{43})A_{4}
\end{array} 
\end{equation}

\subsubsection{Drugi model sekwencyjny}

\begin{equation} \label{eq:sek2}
\begin{array}{l}
T_{1} \geq t_{k12} + d_{1}A_{2} \\
T_{2} \geq t_{k12} + t_{k23} + max(d_{211}A_{2}, t_{k34} + d_{431}A_{4}) + t_{k24} + d_{212}A_{2} \\
T_{3} \geq t_{k12} + t_{k23} + t_{k34} + d_{3}A_{3} \\
T_{4} \geq t_{k12} + t_{k23} + max(d_{211}A_{2}, t_{k34} + d_{431}A_{4}) + t_{k24} + (d_{432} + d_{42})A_{4}
\end{array} 
\end{equation}

\subsubsection{Trzeci model sekwencyjny}

\begin{equation} \label{eq:sek3}
\begin{array}{l}
T_{1} \geq t_{k12} + d_{1}A_{2} \\
T_{2} \geq t_{k12} + t_{k23} + t_{k24} + d_{2}A_{2} \\
T_{3} \geq t_{k12} + t_{k23} + max(d_{321}A_{3}, t_{k24} + d_{421}A_{4}) + t_{k34} + d_{322}A_{3} \\
T_{4} \geq t_{k12} + t_{k23} + max(d_{321}A_{3}, t_{k24} + d_{421}A_{4}) + t_{k34} + (d_{422} + d_{43})A_{4}
\end{array}
\end{equation}

\subsection{Model równoległy}

Model równoległy można przedstawić z wykorzystaniem zmiennyj binarnej $bg24$.

\begin{equation} \label{eq:bg24}
\begin{array}{l}
t_{k24} > t_{k23} + t_{k34} \iff bg24 = 1 \\
t_{k24} \leq t_{k23} + t_{k34} \iff bg24 = 0
\end{array}
\end{equation}

Powyższe fakty można zapisać w następującej postaci:

\begin{equation}
\begin{array}{l}
bg24 < 1 \\
Mbg24 \geq t_{k24} - t_{k23} - t_{k34} \\
M(1 - bg24) > t_{k23} + t_{k34} - t_{k24}
\end{array}
\end{equation}

Na procesorze $P1$ obliczenia rozpoczynają się równocześnie z komunikacją $t_{k12}$ od razu po starcie systemu.

\begin{equation}
\begin{array}{l}
T_{1} \geq d_{1}A_{1} \\
T_{1} \geq t_{k12}
\end{array}
\end{equation}

Na procesorze $P2$ obliczenia rozpoczynają się po zakończeniu komunikacji $t_{k12}$.
W tym samym momencie rozpoczyna się komunikacja $t_{k23}$ oraz $t_{k24}$.

\begin{equation}
\begin{array}{l}
T_{2} \geq t_{k12} + d_{2}A_{2} \\
T_{2} \geq t_{k12} + max(t_{k23}, t_{k24})
\end{array}
\end{equation}

Na procesorze $P3$ obliczenie i komunikacja rozpoczynają się równolegle po zakończeniu komunikacji $t_{k23}$.

\begin{equation}
\begin{array}{l}
T_{3} \geq t_{k12} + t_{k23} + d_{3}A_{3} \\
T_{3} \geq t_{k12} + t_{k23} + t_{k34}
\end{array}
\end{equation}

Na procesorze $P4$ sytuacja jest bardziej złożona. W zależności kolejności otrzymania danych procesor ten rozpocznie stosowne obliczenia.
Warto zauważyć, że $P4$ nie wysyła danych, a jedynie odbiera je.
Jeśli zajdzie warunek $t_{k24} \leq t_{k23} + t_{k34}$ to zastosowane zostanie równanie (\ref{eq:row41}), w przeciwnym wypadku równanie (\ref{eq:row42}).

\begin{equation} \label{eq:row41}
\begin{array}{rl}
T_{4} \geq &t_{k12} + max(t_{k24} + d_{421}A_{4}, t_{k23} + t_{k34}) \\
&+ (d_{422} + d_{43})A_{4} - Mbg24 \\
\end{array}
\end{equation}

\begin{equation} \label{eq:row42}
\begin{array}{rl}
T_{4} \geq &t_{k12} + max(t_{k23} + t_{k34} + d_{431}A_{4}, t_{k24}) \\
&+ (d_{432} + d_{42})A_{4} - (1-bg24)M
\end{array}
\end{equation}
