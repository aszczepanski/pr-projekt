
\subsection{Wersja 4}

\subsubsection{Opis rozwiązania}

Jest to rozszerzona wersja 3 o równoległe z obliczeniami pobranie kolejnych danych (na poziomie bloku). Ma to spowodować złagodzenie kosztów synchronizacji.

\begin{enumerate}[(a)]

\item \textbf{Pobranie do rejestru} \newline

\lstinputlisting[caption=Mnożenie macierzy kwadratowych na GPU -- wersja 4 z pobraniem do rejestru.]{./code/matrix_multiplication_4a.cpp}

\item \textbf{Pobranie do pamięci współdzielonej} \newline

\lstinputlisting[caption=Mnożenie macierzy kwadratowych na GPU -- wersja 4 z pobraniem do pamięci współdzielonej.]{./code/matrix_multiplication_4b.cpp}

\end{enumerate}

\subsubsection{Teoretyczna zajętość SM}

\begin{enumerate}[(a)]

\item \textbf{Pobranie do rejestru} \newline

\begin{center}
\begin{table}[H]
\centering
\begin{tabular}{|c|c|c|c|c|}
\hline
\multicolumn{2}{|c|}{\multirow{2}{*}{Kryterium}} & \multicolumn{2}{c|}{Teoretyczna wartość} & \multirow{2}{*}{Limit GPU} \\ \cline{3-4}
\multicolumn{2}{|c|}{} & 8x8 & 16x16 & \\ \hline
\multirow{4}{*}{Zajętość SM} & Aktywne bloki & 8 & 4 & 8 \\ \cline{2-5}
& Aktywne warpy & 16 & 32 & 32 \\ \cline{2-5}
& Aktywne wątki & 512 & 1024 & 1024 \\ \cline{2-5}
& Zajętość & 50\% & 100\% & 100\% \\ \hline
\multirow{3}{*}{Warpy} & Wątki/Blok & 64 & 256 & 512 \\ \cline{2-5}
& Warpy/Blok & 2 & 8 & 16 \\ \cline{2-5}
& Limit bloków & 16 & \textcolor{red}{4} & 8 \\ \hline
\multirow{3}{*}{Rejestry} & Rejestry/Wątek & 13 & 13 & 128 \\ \cline{2-5}
& Rejestry/Blok & 1024 & 3584 & 16384 \\ \cline{2-5}
& Limit bloków & 16 & \textcolor{red}{4} & 8 \\ \hline
\multirow{2}{*}{Pamięć współdzielona} & Pamięć współdzielona/Blok & 556 & 2092 & 16384 \\ \cline{2-5}
& Limit bloków & 16 & 6 & 8 \\ \hline
\end{tabular}
\caption{Teoretyczna zajętość SM -- wersja 4. z pobraniem do rejestru.}
\end{table}
\end{center}

Podobnie jak dla poprzednich wersji, dla bloku 8x8 limitem okazuje się być maksymalna ilość bloków na SM, stąd zajętość $ 16 / 32 = 50\% $. \\
Dla macierzy 16x16 limitem są warpy i rejestry. Przypada odpowiednio $ 8 $ warpów na blok, co daje limit $ 4 $ aktywnych bloków. $ 3584 $ rejestrów na blok również daje limit $ 4 $ aktywnych bloków. Zajętość dla bloku 16x16 wynosi $ 100\% $. \\

\item \textbf{Pobranie do pamięci współdzielonej} \newline

\begin{center}
\begin{table}[H]
\centering
\begin{tabular}{|c|c|c|c|c|}
\hline
\multicolumn{2}{|c|}{\multirow{2}{*}{Kryterium}} & \multicolumn{2}{c|}{Teoretyczna wartość} & \multirow{2}{*}{Limit GPU} \\ \cline{3-4}
\multicolumn{2}{|c|}{} & 8x8 & 16x16 & \\ \hline
\multirow{4}{*}{Zajętość SM} & Aktywne bloki & 8 & 3 & 8 \\ \cline{2-5}
& Aktywne warpy & 16 & 24 & 32 \\ \cline{2-5}
& Aktywne wątki & 512 & 768 & 1024 \\ \cline{2-5}
& Zajętość & 50\% & 75\% & 100\% \\ \hline
\multirow{3}{*}{Warpy} & Wątki/Blok & 64 & 256 & 512 \\ \cline{2-5}
& Warpy/Blok & 2 & 8 & 16 \\ \cline{2-5}
& Limit bloków & 16 & 4 & 8 \\ \hline
\multirow{3}{*}{Rejestry} & Rejestry/Wątek & 14 & 14 & 128 \\ \cline{2-5}
& Rejestry/Blok & 1024 & 3584 & 16384 \\ \cline{2-5}
& Limit bloków & 16 & 4 & 8 \\ \hline
\multirow{2}{*}{Pamięć współdzielona} & Pamięć współdzielona/Blok & 1068 & 4140 & 16384 \\ \cline{2-5}
& Limit bloków & 16 & \textcolor{red}{3} & 8 \\ \hline
\end{tabular}
\caption{Teoretyczna zajętość SM -- wersja 4. z pobraniem do pamięci współdzielonej.}
\end{table}
\end{center}

Podobnie jak dla poprzednich wersji, dla bloku 8x8 limitem okazuje się być maksymalna ilość bloków na SM, stąd zajętość $ 16 / 32 = 50\% $. \\
Dla macierzy 16x16 limitem jest pamięć współdzielona. $4140$ bajtów na blok daje limit $3$ aktywnych bloków. W porównaniu z poprzednimi wersjami zajętość SM dla bloku 16x16 spada i wynosi $ 75\% $. \\

\end{enumerate}

\subsubsection{Wyniki pomiarów}

\begin{enumerate}[(a)]

\item \textbf{Pobranie do rejestru} \newline

\begin{enumerate}[1.]

\item \textbf{Czas trwania obliczeń} \newline

\item \textbf{Ilość operacji zmiennoprzecinkowych na sekundę} \newline

\item \textbf{Ilość instrukcji na sekundę} \newline

\item \textbf{CGMA} \newline

\end{enumerate}

\item \textbf{Pobranie do pamięci współdzielonej} \newline

\begin{enumerate}[1.]

\item \textbf{Czas trwania obliczeń} \newline

\item \textbf{Ilość operacji zmiennoprzecinkowych na sekundę} \newline

\item \textbf{Ilość instrukcji na sekundę} \newline

\item \textbf{CGMA} \newline

\end{enumerate}

\end{enumerate}
