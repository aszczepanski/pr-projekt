
\section{Model matematyczny}

Wszystkie zmienne wykorzystywane poniżej są zdefiniowane w tabeli \ref{tab:notacja}.

\begin{table}[!ht]
\centering
\begin{tabular}{|c|l|}
\hline
$T$ & minimalizowany czas zakończenia wszystkich dzialań \\ \hline
$T_{a}$ & najpóźniejszy czas zakończenia działań na procesorze $a$ \\ \hline
$A_{a}$ & czas potrzebny na obliczenie jednej porcji danych na procesorze $a$ \\ \hline
$S_{ab}$ & czas potrzebny na zainicjowanie połączenia między procesorami $a$ i $b$ \\ \hline
$C_{ab}$ & czas potrzebny na przesłanie jednej porcji danych na łączu między procesorami $a$ i $b$ \\ \hline
$V$ & suma wszystkich danych do obliczenia \\ \hline
$d_{a}$ & dane obliczane na procesorze $a$ \\ \hline
$d_{ab}$ & dane obliczne na procesorze $a$ otrzymane z procesora $b$ \\ \hline
$d_{abx}$ & część $x$ danych obliczanych na procesorze $a$ otrzymanych z procesora $b$ ($x = 1, 2$) \\ \hline
$t_{kab}$ & czas komunikacji na łączu pomiędzy procesorami $a$ i $b$ \\ \hline
$x_{ab}$ & zmienna binarna opisana w równaniu (\ref{eq:x}) \\ \hline
$bg24$ & zmienna binarna opisana w równaniu (\ref{eq:bg24}) \\ \hline
$M$ & duża liczba wyznaczona w równaniu (\ref{eq:M}) \\ \hline
\end{tabular}
\caption{Notacja wykorzystywana w raporcie}
\label{tab:notacja}
\end{table}

\subsection{Wzory wspólne dla wszystkich modeli}

We wszystkich modelach minimalizujemy najdłuższy czas zakończenia pracy procesora.

\begin{equation} \label{eq:min}
\begin{array}{l}
min T \\
T \geq T_{1} \\
T \geq T_{2} \\
T \geq T_{3} \\
T \geq T_{4}
\end{array}
\end{equation}

Istotne są założenia wynikające z definicji zmiennych dotyczących ilości danych.

\begin{equation} \label{eq:data}
\begin{array}{l}
d_{1} + d_{2} + d_{3} + d_{4} = V \\
d_{211} + d_{212} = d_{2} \\
d_{321} + d_{322} = d_{3} \\
d_{43} + d_{42} = d_{4} \\
d_{421} + d_{422} = d_{42} \\
d_{431} + d_{432} = d_{43}
\end{array}
\end{equation}

Dla wszystkich modeli można zastosować te same wzory na czas komunikacji na poszczególnych łączach.
Łączem $P1 \to P2$ należy przesłać wszystkie dane nieprzetwarzane przez $P1$.
Łączem $P2 \to P3$ należy przesłać dane obliczane przez procesor $P3$ oraz te, które będą przesyłane na łączu $P3 \to P4$. 

\begin{equation} \label{eq:tk}
\begin{array}{l}
t_{k12} = S_{12}x_{12} + (V - d_{1})C_{12} \\
t_{k23} = S_{23}x_{23} + C_{23}d_{3} + C_{23}d_{43} \\
t_{k24} = S_{24}x_{24} + C_{24}d_{42} \\
t_{k34} = S_{34}x_{34} + C_{34}d_{43}
\end{array}
\end{equation}

W powyższych wzorach wykorzystywane są zmienne binarne. Są one tak dobrane, że jeśli między procesorami $a$ i $b$ występuje komunikacja,
to $x_{ab} = 1$, w przeciwnym wypadku solver może przypisać im wartość $0$ w celu optymalizacji rozwiązania.

\begin{equation} \label{eq:x}
\begin{array}{l}
x_{12} \leq 1 \\
Vx_{12} \geq d_{2} + d_{3} + d_{4} \\
x_{23} \leq 1 \\
Vx_{23} \geq d_{3} + d_{43} \\
x_{24} \leq 1 \\
Vx_{24} \geq d_{42} \\
x_{34} \leq 1 \\
V_{x34} \geq d_{43}
\end{array}
\end{equation}

Istotne jest, żeby zmienne dotyczące ilości danych oraz zmienne binarne były całkowite. \\

Dla analizowanej architektury zawsze da się znaleźć rozwiązanie, które jest niegorsze od
przetworzenia wszystkich danych przez procesor $P1$ bez komunikacji. Można zatem przyjąć:

\begin{equation} \label{eq:M}
M = d_{1}A_{1}
\end{equation}

W poniższych wzorach stosowana jest konstrukcja:

\begin{equation} \label{eq:max1}
T_{x} \geq Y + max(a, b)
\end{equation}

Jest to zabieg mający na celu zwiększenie czytelności i może być zastąpiony 2 nierównościami:

\begin{equation} \label{eq:max2}
\begin{array}{l}
T_{x} \geq Y + a \\
T_{x} \geq Y + b
\end{array}
\end{equation}

Solver zawsze wybierze silniejsze ograniczenie.

\subsection{Modele sekwencyjne}

\subsubsection{Pierwszy model sekwencyjny}

\begin{equation} \label{eq:sek1}
\begin{array}{l}
T_{1} \geq t_{k12} + d_{1}A_{2} \\
T_{2} \geq t_{k12} + t_{k24} + t_{k23} + d_{2}A_{2} \\
T_{3} \geq t_{k12} + t_{k24} + max(t_{k23} + d_{321}A_{3}, d_{421}A_{4}) + t_{k34} + d_{322}A_{3} \\
T_{4} \geq t_{k12} + t_{k24} + max(t_{k23} + d_{321}A_{3}, d_{421}A_{4}) + t_{k34} + (d_{422} + d_{43})A_{4}
\end{array} 
\end{equation}

W tym przypadku problematyczna jest konieczność synchronizacji procesorów $P3$ i $P4$ przed rozpoczęciem komunikacji $P3 \to P4$.
Obliczane dane na procesorze $P3$ zostają rozbite na 2 części ($d_{321}$ i $d_{322}$).
Podobnie dane obliczne na procesorze $P4$ zostają rozbite na 2 części ($d_{421}$ i $(d_{422} + d_{43})$).
Komunikacja $P2 \to P3$ i obliczenia $d_{321}$ występują w tym samym czasie co obliczenia $d_{421}$,
więc w przypadku zarówno $T_{3}$ jak i $T_{4}$ występuje człon $max(t_{k23} + d_{321}A_{3}, d_{421}A_{4})$. \\

Warto przeanalizować najciekawsze przypadki w których któraś z komunikacji nie występuje. \\

Jeśli nie wystąpi komunikacja $P1 \to P2$ to:

\begin{equation}
\begin{array}{l}
t_{k12} = t_{k24} = t_{k23} = t_{k34} \\
d_{1} = V \\
d_{2} = d_{3} = d_{4} = 0 \\
T_{1} \geq d_{1}A_{2} \\
T_{2} \geq 0 \\
T_{3} \geq 0 \\
T_{4} \geq 0
\end{array} 
\end{equation}

Jeśli nie wystąpi komunikacja $P2 \to P4$ i $P2 \to P3$ to:
\begin{equation}
\begin{array}{l}
t_{k24} = t_{k23} = t_{k34} = 0 \\
d_{3} = d_{4} = 0 \\
T_{1} \geq t_{k12} + d_{1}A_{2} \\
T_{2} \geq t_{k12} + d_{2}A_{2} \\
T_{3} \geq t_{k12} \\
T_{4} \geq t_{k12}
\end{array}
\end{equation}

Z powyższych wzorów łatwo wywnioskować, że $T_{3}$ i $T_{4}$ są nieprawdziwe. Rzeczywiście jest to prawda, ale ich wartości nie zmieniają rozwiązania ostatecznego.
Przypadek ten pokazuje, że istotne jest, żeby poprawnie interpretować dane otrzymywane z solvera. W każdym przypadku zwróci on poprawny czas zakończenia działania całego systemu $T$
oraz poprawne ilości danych obliczane przez poszczególne procesory. \\

Jeśli nie wystąpi komunikacja $P2 \to P4$ to:
\begin{equation}
\begin{array}{l}
t_{k24} = 0 \\
d_{42} = 0 \\
T_{1} \geq t_{k12} + d_{1}A_{2} \\
T_{2} \geq t_{k12} + t_{k23} + d_{2}A_{2} \\
T_{3} \geq t_{k12} + t_{k23} + d_{321}A_{3} + t_{k34} + d_{322}A_{3} \\
T_{4} \geq t_{k12} + t_{k23} + d_{321}A_{3} + t_{k34} + d_{43}A_{4}
\end{array}
\end{equation}

W tym przypadku solver optymalizując może doprowadzić do sytuacji:

\begin{equation}
\begin{array}{l}
T_{3} \geq t_{k12} + t_{k23} + t_{k34} + d_{3}A_{3} \\
T_{4} \geq t_{k12} + t_{k23} + t_{k34} + d_{4}A_{4}
\end{array}
\end{equation}

W pozostałych dwóch modelach ze wszystkimi problemami poradziliśmy sobie analogicznie, dlatego przytaczamy jedynie odpowiednie wzory.

\subsubsection{Drugi model sekwencyjny}

\begin{equation} \label{eq:sek2}
\begin{array}{l}
T_{1} \geq t_{k12} + d_{1}A_{2} \\
T_{2} \geq t_{k12} + t_{k23} + max(d_{211}A_{2}, t_{k34} + d_{431}A_{4}) + t_{k24} + d_{212}A_{2} \\
T_{3} \geq t_{k12} + t_{k23} + t_{k34} + d_{3}A_{3} \\
T_{4} \geq t_{k12} + t_{k23} + max(d_{211}A_{2}, t_{k34} + d_{431}A_{4}) + t_{k24} + (d_{432} + d_{42})A_{4}
\end{array} 
\end{equation}

\subsubsection{Trzeci model sekwencyjny}

\begin{equation} \label{eq:sek3}
\begin{array}{l}
T_{1} \geq t_{k12} + d_{1}A_{2} \\
T_{2} \geq t_{k12} + t_{k23} + t_{k24} + d_{2}A_{2} \\
T_{3} \geq t_{k12} + t_{k23} + max(d_{321}A_{3}, t_{k24} + d_{421}A_{4}) + t_{k34} + d_{322}A_{3} \\
T_{4} \geq t_{k12} + t_{k23} + max(d_{321}A_{3}, t_{k24} + d_{421}A_{4}) + t_{k34} + (d_{422} + d_{43})A_{4}
\end{array}
\end{equation}

\subsection{Model równoległy}

Model równoległy można przedstawić z wykorzystaniem zmiennej binarnej $bg24$.

\begin{equation} \label{eq:bg24}
\begin{array}{l}
t_{k24} > t_{k23} + t_{k34} \Rightarrow bg24 = 1 \\
t_{k24} < t_{k23} + t_{k34} \Rightarrow bg24 = 0
\end{array}
\end{equation}

Jeśli $t_{k24} = t_{k23} + t_{k34}$ solver może przypisać zmiennej $bg24$ wartość $0$ lub $1$.
Powyższe fakty można zapisać w następującej postaci:

\begin{equation}
\begin{array}{l}
bg24 \leq 1 \\
Mbg24 \geq t_{k24} - t_{k23} - t_{k34} \\
M(1 - bg24) \geq t_{k23} + t_{k34} - t_{k24}
\end{array}
\end{equation}

Na procesorze $P1$ obliczenia rozpoczynają się równocześnie z komunikacją $t_{k12}$ od razu po starcie systemu.

\begin{equation}
\begin{array}{l}
T_{1} \geq d_{1}A_{1} \\
T_{1} \geq t_{k12}
\end{array}
\end{equation}

Na procesorze $P2$ obliczenia rozpoczynają się po zakończeniu komunikacji $t_{k12}$.
W tym samym momencie rozpoczyna się komunikacja $t_{k23}$ oraz $t_{k24}$.

\begin{equation}
\begin{array}{l}
T_{2} \geq t_{k12} + d_{2}A_{2} \\
T_{2} \geq t_{k12} + max(t_{k23}, t_{k24})
\end{array}
\end{equation}

Na procesorze $P3$ obliczenie i komunikacja rozpoczynają się równolegle po zakończeniu komunikacji $t_{k23}$.

\begin{equation}
\begin{array}{l}
T_{3} \geq t_{k12} + t_{k23} + d_{3}A_{3} \\
T_{3} \geq t_{k12} + t_{k23} + t_{k34}
\end{array}
\end{equation}

Na procesorze $P4$ sytuacja jest bardziej złożona. W zależności kolejności otrzymania danych procesor ten rozpocznie stosowne obliczenia.
Warto zauważyć, że $P4$ nie wysyła danych, a jedynie odbiera je.
Jeśli zajdzie warunek $t_{k24} \leq t_{k23} + t_{k34}$ to zastosowane zostanie równanie (\ref{eq:row41}), w przeciwnym wypadku równanie (\ref{eq:row42}).

\begin{equation} \label{eq:row41}
\begin{array}{rl}
T_{4} \geq &t_{k12} + max(t_{k24} + d_{421}A_{4}, t_{k23} + t_{k34}) \\
&+ (d_{422} + d_{43})A_{4} - Mbg24 \\
\end{array}
\end{equation}

\begin{equation} \label{eq:row42}
\begin{array}{rl}
T_{4} \geq &t_{k12} + max(t_{k23} + t_{k34} + d_{431}A_{4}, t_{k24}) \\
&+ (d_{432} + d_{42})A_{4} - (1-bg24)M
\end{array}
\end{equation}

\newpage
