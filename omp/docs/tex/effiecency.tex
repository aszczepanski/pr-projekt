\section{Pomiary efektywności}

\subsection{Przyśpieszenie obliczeń równoległych}

\subsubsection{Pomiary}

\begin{table}[!ht]
\caption{Porównanie szybkości wybranych algorytmów wobec $sum\_ij$}
\begin{tabular}{|c|c|c|}

\hline
  Algorytm &
  Czas wykonania w $ms$ &
  Przyśpieszenie względem $sum\_ij$ \\

\hline
  $sum\_ij$ &
  253 &
  1.000 \\

\hline
  $sum\_ji$ &
  2362 &
  0.107 \\

\hline
  $sum\_par\_ij$ &
  87 &
  2.908 \\

\hline
  $sum\_par\_ji$ &
  1189 &
  0.213 \\

\hline

\end{tabular}
\end{table}

Warto zwrócić uwagę na fakt, że oprócz braku lub obecności zrównoleglenia znaczący wpływ na szybkość przetwarzania ma kolejność uszeregowania pętli. Zostało to szerzej omówione w sekcji \ref{sec:trafienia}. Przyśpieszenia dla poszczególnych uszeregowań pętli prezentują się następująco:

\begin{table}[!ht]
\caption{Porównanie szybkości wybranych algorytmów (o kolejności pętli $ij$) wobec $sum\_ij$}
\begin{tabular}{|c|c|c|}

\hline
  Algorytm &
  Czas wykonania w $ms$ &
  Przyśpieszenie względem $sum\_ij$ \\

\hline
  $sum\_ij$ &
  2362 &
  1.000 \\

\hline
  $sum\_par\_ij$ &
  87 &
  2.908 \\

\hline

\end{tabular}
\end{table}

\begin{table}[!ht]
\caption{Porównanie szybkości wybranych algorytmów (o kolejności pętli $ji$) wobec $sum\_ji$}
\begin{tabular}{|c|c|c|}

\hline
  Algorytm &
  Czas wykonania w $ms$ &
  Przyśpieszenie względem $sum\_ji$ \\

\hline
  $sum\_ji$ &
  2362 &
  1.000 \\

\hline
  $sum\_par\_ji$ &
  1189 &
  1.987 \\

\hline

\end{tabular}
\end{table}

\subsubsection{Podsumowanie}

Zrównoleglenie przetwarzania znacząco przyśpiesza jego czas przetwarzania. Znaczeny wpływ na wielkość wartości tego przyśpieszenia ma kolejność uszeregowania pętli. W przybliżeniu jest to:
\begin{itemize}
\item trzykrotne przyśpieszenie dla uszeregowania pętli $ij$
\item dwukrotne przyśpieszenie dla uszeregowania pętli $ji$
\end{itemize}

\subsection{Współczynniki braku trafień do pamięci}
\label{sec:trafienia}


