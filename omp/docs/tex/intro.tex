\section{Wstęp}

\subsection{Opis problemu}

Głównym założeniem projektu było zapoznanie się biblioteką OpenMP na podstawie równoległego sumowania komórek tablicy. W ramach projektu zrealizowaliśmy 2 algorytmy sekwencyjne oraz 2 algorytmy zrównleglone. Celem dwóch różnych algorytmów sekwencyjnych było zbadanie wpływu sekcyjności pamięci podręcznej na czas realizacji zadania. W przypadku algorytmów zrównoleglonych badaliśmy wpływ kolejności uszeregowania pętli na końcowy rezultat.

\subsection{Punkt odniesienia (algorytm sekwencyjny - kolejność ij)}

\lstinputlisting{./code/sum_ij.cpp}


\subsection{Badane algorytmy}


\subsubsection{Algorytm sekwencyjny - kolejność ji}

\lstinputlisting{./code/sum_ji.cpp}


\subsubsection{Algorytm zrównoleglony - kolejność ij}

\lstinputlisting{./code/sum_par_ij.cpp}


\subsubsection{Algorytm zrównoleglony - kolejność jj}

\lstinputlisting{./code/sum_par_ji.cpp}


\subsubsection{Algorytm na sekcyjność pamięci}

\lstinputlisting{./code/sum_sec.cpp}


\subsubsection{Algorytm na  pobranie  z wyprzedzeniem}

\lstinputlisting{./code/sum_pf.cpp}

