
\subsection{Wersja 5}

\subsubsection{Opis rozwiązania}

Ostatnia wersja rozszerza wersję 4 -- każdy wątek wykonuje większą pracę. Zostało to zrealizowane przez zwiększenie pobieranych i obliczanych danych z jednej do czterech.

\lstinputlisting[caption=Mnożenie macierzy kwadratowych na GPU -- wersja 5.]{./code/matrix_multiplication_5.cpp}

\subsubsection{Teoretyczna zajętość SM}

\subsubsection{Wyniki pomiarów}

\paragraph{Czas trwania obliczeń}

\begin{table}[H]
\centering
\begin{tabular}{|c|c|c|}
\hline
\multirow{2}{*}{Rozmiar macierzy} & \multicolumn{2}{c|}{Rozmiar bloku} \\ \cline{2-3}
& 8x8 & 16x16 \\ \hline
128x128 & 0.118 & 0.139 \\ \hline
256x256 & 0.877 & 0.763 \\ \hline
384x384 & 2.622 & 2.287 \\ \hline
512x512 & 6.248 & 5.471 \\ \hline
640x640 & 12.047 & 10.632 \\ \hline
\end{tabular}
\caption{Czas obliczeń [ms] -- wersja 3.}
\end{table}

\paragraph{Ilość operacji zmiennoprzecinkowych na sekundę}

\begin{table}[H]
\centering
\begin{tabular}{|c|c|c|}
\hline
\multirow{2}{*}{Rozmiar macierzy} & \multicolumn{2}{c|}{Rozmiar bloku} \\ \cline{2-3}
& 8x8 & 16x16 \\ \hline
128x128 & 35.396 & 30.229 \\ \hline
256x256 & 38.248 & 43.995 \\ \hline
384x384 & 43.189 & 49.512 \\ \hline
512x512 & 42.966 & 49.063 \\ \hline
640x640 & 43.520 & 49.314 \\ \hline
\end{tabular}
\caption{Ilosc operacji zmiennoprzecinkowych na sekundę (GFLOPS) -- wersja 1.}
\end{table}

\paragraph{Ilość instrukcji na sekundę}

\begin{table}[H]
\centering
\begin{tabular}{|c|c|c|c|}
\hline
\multirow{2}{*}{Rozmiar macierzy} & \multicolumn{2}{c|}{Rozmiar bloku} \\ \cline{2-3}
& 8x8 & 16x16 \\ \hline
128x128 & 0.1481 & 0.1856 \\ \hline
256x256 & 0.1423 & 0.1641 \\ \hline
384x384 & 0.1543 & 0.1951 \\ \hline
512x512 & 0.1574 & 0.1894 \\ \hline
640x640 & 0.1538 & 0.1909 \\ \hline
\end{tabular}
\caption{Ilość instrukcji wykonana na sekundę (GIPS) -- wersja 1.}
\end{table}

\paragraph{CGMA}

\begin{table}[H]
\centering
\begin{tabular}{|c|c|c|c|}
\hline
\multirow{2}{*}{Rozmiar macierzy} & \multicolumn{2}{c|}{Rozmiar bloku} \\ \cline{2-3}
& 8x8 & 16x16 \\ \hline
128x128 & 240.941 & 1129.931 \\ \hline
256x256 & 256.250 & 1236.528 \\ \hline
384x384 & 250.776 & 1276.675 \\ \hline
512x512 & 253.050 & 1297.743 \\ \hline
640x640 & 255.393 & 1310.720 \\ \hline
\end{tabular}
\caption{Stosunek ilości operacji zmiennoprzecinkowych do ilości operacji odczytu/zapisu z pamięci globalnej -- wersja 1.}
\end{table}

