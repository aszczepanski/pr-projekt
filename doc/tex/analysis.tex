
\section{Analiza systemu}

Celem projektu była analiza możliwości przetwarzania równoległego w architekturze czwórki z początkowymi danymi w dolnym węźle P1 (jak pokazano na rysunku \ref{fig:schema}).
W ramach projektu zajmowaliśmy się dwoma modelami - sekwencyjnym oraz równoległym z jedną komunikacją na łączu.

\begin{figure}[!ht]
\centering
\includegraphics[width=0.5\textwidth]{schema.pdf}
\caption{Schemat systemu wraz z oznaczeniem procesorów}
\label{fig:schema}
\end{figure}

Biorąc pod uwagę fakt, że węzeł P1 jako jedyny  posiada dane w stanie początkowym może on odgrywać jedynie rolę nadawcy danych. Węzeł P2 może zarówno otrzymywać (z węzła P1), jak i nadawać dane (do węzłów P3 i P4). Na łączu 1 komunikacja będzie się odbywać w kierunku “do” węzła P2, natomiast na łączach 2 i 4 będzie się odbywać w kierunku “z” węzła P2. Niestety, nie możemy w modelu ogólnym założyć, czy dana komunikacja wystąpi. Trzeba pamiętać, że komunikacja na łączach 2 i 4 ma szanse wystąpić tylko jeśli wystąpi komunikacja na łączu 1 (węzeł P2 otrzyma dane z węzła P1). Komunikacja na łączu 1 (pomiędzy węzłami P1 i P2) nie musi wystąpić jeśli koszt transmisji danych na tym łączy będzie stosunkowo duży. W przypadku węzłów P3 i P4 warto zauważyć symetrię \ref{fig:symetry}. Węzły te mogą pełnić zarówno rolę odbiorcy jak i nadawcy.

\begin{figure}[!ht]
\centering
\includegraphics[width=0.5\textwidth]{symetry.pdf}
\caption{Symetria w analizowanym systemie oznaczona linią przerywaną}
\label{fig:symetry}
\end{figure}
 
\subsection{Analiza modelu sekwencyjnego}

Analiza modelu sekwencyjnego.

\subsubsection{Pierwszy model sekwencyjny}

\subsubsection{Drugi model sekwencyjny}

\subsubsection{Trzeci model sekwencyjny}

\subsection{Analiza modelu równoległego}

Analiza modelu równoległego.

\subsubsection{Pierwszy model równoległy}

\subsubsection{Drugi model równoległy}

