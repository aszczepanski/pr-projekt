\section{Wpływ sekcyjności pamięci}

\subsection{Wstęp}

W trakcie naszych badań sprawdziliśmy wpływ sekcyjności pamięci na czas przetwarzania. Kluczowe dla tego podpunktu algorytmy to:

\begin{itemize}
\item $sum\_ij$ - podstawowy algorytm sekwencyjny, służący jako punkt odeniesienia
\item $sum\_sec$ - algorytm sekwencyjny w którym badaliśmy sekcyjność pamięci
\end{itemize}

W wykorzystywanym systemie długość lini PP wynosi $64B$, a liczba lini w pojedyńczej stronie wynosi $512$.\newline

W metodzie $sum\_sec$ w kolejnych iteracjach odwołujemy się do komórki tabeli z tej samej kolumny, z wiersza przesuniętego o ilość lini w stronie. Powoduje to konieczność wymiany całej lini PP w każdej iteracji.

\subsection{Wyniki}

Zgodnie z założeniem, powinien to być najwolniejszy algorytm, ponieważ spodziewana jest bardzo wysoka liczba odwołań do pamięci RAM or wysoki stosunek braku trafień do pamięci podręcznej.

\begin{table}[H]
\centering
\begin{tabular}{|l|c|c|c|c|}
\hline
Algorytm & Czas [ms] & DC missess & L2 misses & DRAM access \\ \hline
$sum\_ij$ & 253 & 24 & 340 & 352 \\ \hline
$sum\_sec$ & 19552 & 5256 & 11024 & 5920 \\ \hline
\end{tabular}
\caption{Porównanie wybranych algorytmów pod kątem sekcyjności pamięci}
\end{table}

Wyniki przerosły nasze oczekiwania - $sum\_sec$ wykonał się ponad $77$ razy wolniej. Pokazuje to jest wielkie znaczenie na szybkość realizacji przetwarzania ma zachowanie cechy lokalności przestrzennej. Wyraźnie widać, że w przypadku algorytmu użytego jako punkt odniesienia liczba braków trafień do pamięci podręcznej L1 była znacznie mniejsza. Sytuacja wygląda bardzo podobnie dla pamięci podręcznej L2. Konsekwencją braku trafień do pamięci podręcznej jest konieczność większej liczby pobrań danych z pamięci dynamicznej co odzwierciedla współczynnik $DRAM\_accesses$.
