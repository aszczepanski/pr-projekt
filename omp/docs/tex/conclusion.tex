\section{Podsumowanie}

Zrealizowane zadania pozowliły nam na lepsze poznanie przetwarzania równoległwego. Uświadomiły nas o korzyściach płynących z jego stosowania. Dwa główne wnioski jakie można wyciągnąć to:

\begin{itemize}
\item przetwarzanie równoległe jest szybsze od sekwencyjnego (doskonale widać poprzez porównanie algorytmów $sum\_ij$ oraz $sum\_par\_ij$)
\item ważne podczas pisania kodu jest zaplanowanie dostępów do pamięci, zapewnienie własności lokalności przestrzennej w znaczący sposób przyśpiesza działanie progoramu ($sum\_ij$ 77 krotnie szybszy od $sum\_sec$)
\end{itemize}

Najszybszy algorytm ($sum\_par\_ij$) był blisko 225 razy szybszy od najwolniejszego ($sum\_sec$). Było dla nas sporym zaskoczeniem, że takie różnice można uzyskać dla wydawało by się zwykłego sumowania komórek z tablicy.\newline

Dodatkowo warto zauważyć, że rozmiar danych liniowo (w naszym przypadku) ogranicza czas wykonania programu niezależnie czy obliczenia są wykonywane równolegle czy sekwencyjnie.
