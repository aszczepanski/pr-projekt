\documentclass{article}

\usepackage{polski}
\usepackage{setspace}
\usepackage[polish]{babel}
\usepackage[utf8]{inputenc}

\usepackage{graphicx}
\graphicspath{{./img/}}

\usepackage{listings}
\usepackage{enumerate}
\usepackage{float}
\usepackage{datetime}
\usepackage{amsmath}
\usepackage{color}
\usepackage{subfig}
\usepackage{multirow}
\usepackage{colortbl}
\usepackage{tikz}
\usetikzlibrary{plotmarks}
\usepackage{pgfplots}
\pgfplotsset{compat=newest}
\pgfplotsset{major grid style={dashed,gray}}
\pgfplotsset{minor grid style={dotted,gray}}

\definecolor{dkgreen}{rgb}{0,0.6,0}
\definecolor{gray}{rgb}{0.5,0.5,0.5}
\definecolor{mauve}{rgb}{0.58,0,0.82}

% \renewcommand\lstlistingname{Listing}
\renewcommand{\lstlistlistingname}{Kody źródłowe}

\lstset{%
  frame=none,
  language=C++,
  aboveskip=3mm,
  belowskip=3mm,
  showstringspaces=false,
  columns=fullflexible,
  basicstyle={\small\ttfamily},
  numbers=none,
  numberstyle=\tiny\color{gray},
  keywordstyle=\color{blue},
  commentstyle=\color{dkgreen},
  stringstyle=\color{mauve},
  breaklines=true,
  breakatwhitespace=true
  tabsize=2,
  captionpos=b
}

\lstset{literate=
  {ł}{\l}1 {ę}{\k{e}}1 {ą}{\k{e}}1
  {á}{{\'a}}1 {é}{{\'e}}1 {í}{{\'i}}1 {ó}{{\'o}}1 {ú}{{\'u}}1
  {Á}{{\'A}}1 {É}{{\'E}}1 {Í}{{\'I}}1 {Ó}{{\'O}}1 {Ú}{{\'U}}1
  {à}{{\`a}}1 {è}{{\'e}}1 {ì}{{\`i}}1 {ò}{{\`o}}1 {ù}{{\`u}}1
  {À}{{\`A}}1 {È}{{\'E}}1 {Ì}{{\`I}}1 {Ò}{{\`O}}1 {Ù}{{\`U}}1
  {ä}{{\"a}}1 {ë}{{\"e}}1 {ï}{{\"i}}1 {ö}{{\"o}}1 {ü}{{\"u}}1
  {Ä}{{\"A}}1 {Ë}{{\"E}}1 {Ï}{{\"I}}1 {Ö}{{\"O}}1 {Ü}{{\"U}}1
  {â}{{\^a}}1 {ê}{{\^e}}1 {î}{{\^i}}1 {ô}{{\^o}}1 {û}{{\^u}}1
  {Â}{{\^A}}1 {Ê}{{\^E}}1 {Î}{{\^I}}1 {Ô}{{\^O}}1 {Û}{{\^U}}1
  {œ}{{\oe}}1 {Œ}{{\OE}}1 {æ}{{\ae}}1 {Æ}{{\AE}}1 {ß}{{\ss}}1
  {ç}{{\c c}}1 {Ç}{{\c C}}1 {ø}{{\o}}1 {å}{{\r a}}1 {Å}{{\r A}}1
  {€}{{\EUR}}1 {£}{{\pounds}}1
}



\setlength\parindent{0pt}

\usepackage{fancyhdr}
\pagestyle{fancy}
\renewcommand{\sectionmark}[1]{\markboth{#1}{}}
\renewcommand{\subsectionmark}[1]{\markright{#1}{}}
\fancyfoot[C]{Strona \thepage}
\fancypagestyle{plain}{%
  \fancyhead{} % get rid of headers
  \renewcommand{\headrulewidth}{0pt} % and the line
}

\usepackage{titlesec}

\begin{document}

\begin{titlepage}

\newcommand{\HRule}{\rule{\linewidth}{0.5mm}} % Defines a new command for the horizontal lines, change thickness here

\center % Center everything on the page
 
%----------------------------------------------------------------------------------------
%	HEADING SECTIONS
%----------------------------------------------------------------------------------------

\textsc{\LARGE Politechnika Poznańska}\\[1.5cm] % Name of your university/college
\textsc{\Large Wydział Informatyki}\\[0.5cm] % Major heading such as course name
\textsc{\large Przetwarzanie Równoległe}\\[0.5cm] % Minor heading such as course title

%----------------------------------------------------------------------------------------
%	TITLE SECTION
%----------------------------------------------------------------------------------------

\HRule \\[0.4cm]
{ \huge \bfseries Optymalizacja przetwarzania równoległego}\\[0.4cm] % Title of your document
\HRule \\[1.5cm]
 
%----------------------------------------------------------------------------------------
%	AUTHOR SECTION
%----------------------------------------------------------------------------------------

\begin{minipage}{0.4\textwidth}
\begin{flushleft} \large
\emph{Autorzy:}\\
Adam \textsc{Szczepański} \\
Mateusz \textsc{Czajka} % Your name
\end{flushleft}
\end{minipage}
~
\begin{minipage}{0.4\textwidth}
\begin{flushright} \large
\emph{Prowadzący:} \\
dr. Rafał \textsc{Walkowiak} % Supervisor's Name
\end{flushright}
\end{minipage}\\[2cm]

%----------------------------------------------------------------------------------------
%	LOGO SECTION
%----------------------------------------------------------------------------------------

\begin{figure}[h]
\centering
\includegraphics[width=0.3\textwidth]{PUT_logo.png}\\[1cm] % Include a department/university logo - this will require the graphicx package
\end{figure}

%----------------------------------------------------------------------------------------
%	DATE SECTION
%----------------------------------------------------------------------------------------

{\large \today}\\[1cm] % Date, change the \today to a set date if you want to be precise

 
%----------------------------------------------------------------------------------------

\vfill % Fill the rest of the page with whitespace

\end{titlepage}


\thispagestyle{plain}
\tableofcontents
\newpage


\section{Informacje o projekcie}

\subsection{Dane autorów}

\begin{center}
\begin{tabular}{l r}
Mateusz Czajka & 106596 \\
Adam Szczepański & 106593
\end{tabular}
\end{center}

\subsection{Historia projektu}

\begin{center}
\begin{enumerate}
\item Opis pierwszej wersji
\end{enumerate}
\end{center}



\section{Wprowadzenie}

Celem projektu było zapoznanie się z możliwościami przetwarzania na kartach graficznych na przykładzie technologi CUDA. \\

Przygotowaliśmy 5 wersji programu, którego zadaniem było mnożenie macierzy kwadratowych o jednakowych wymiarach na GPU. Każda kolejna wersja jest modyfikacją poprzedniej. \\

Efektywność programów zbadaliśmy przy pomocy profilera Nsight dla Visual Studio. Dla każdej instancji mierzyliśmy:
\begin{itemize}
\item czas wykonania
\item ilość operacji zmienno przecinkowych na sekundę (GFLOPS)
\item ilość instrukcji wykonanych na sekundę (GIPS)
\item przepustowość pamięci globalnej
\item stosunek operacji zmienno przecinkowych do ilości operacji odczytu/zapisu z pamięci globalnej (CGMA)
\item zajętość warpami multiprocesorów
\end{itemize}

\newpage


\section{Ocena efektywności przetwarzania}


\subsection{Wersja 1}

W pierwszej wersji programu wykorzystany został tylko jeden blok wątków. Jeśli rozmiar bloku jest równy rozmiarowi macierzy to każdy z wątków oblicza 1 element wyniku. W przeciwnym przypatku każdy z wątków oblicza $$ {\left(\frac{\text{rozmiar macierzy}}{\text{rozmiar bloku}}\right)}^{2} $$ elementów wyniku. Pamięć współdzielona nie jest wykorzystywana.

\lstinputlisting[caption=Mnożenie macierzy kwadratowych na GPU -- wersja 1.]{./code/matrix_multiplication_1.cpp}

Ponieważ wykorzystany jest tylko jeden blok, obliczenia przeprowadzane są na jednym SM, a co za tym idzie w danej chwili aktywny może być tylko jeden warp. \\
Duży wpływ na prędkość przetwarzania ma wielkość bloku, dla małych bloków efektywność jest bardzo niska -- duża ilość dostępów do pamięci ogranicza prędkość przetwarzania.

\begin{table}[H]
\centering
\begin{tabular}{|c|c|c|c|}
\hline
\multirow{2}{*}{Rozmiar macierzy} & \multicolumn{3}{c|}{Rozmiar bloku} \\ \cline{2-4}
& 8x8 & 16x16 & 22x22 \\ \hline
176x176 & 49.79 & 13.17 & 7.45 \\ \hline
352x352 & 401.35 & 103.85 & 59.70 \\ \hline
528x528 & 1354.40 & 349.77 & 200.36 \\ \hline
\end{tabular}
\caption{Czas obliczeń [ms] -- wersja 1.}
\end{table}



\subsection{Wersja 2}

W drugiej wersji wykorzystywany jest grid wieloblokowy o rozmiarze $$ \frac{\text{rozmiar macierzy}}{\text{rozmiar bloku}} $$
Każdy wątek oblicza jeden element macierzy wynikowej. Pamięć współdzielona nie jest wykorzystywana.

\lstinputlisting[caption=Mnożenie macierzy kwadratowych na GPU -- wersja 2.]{./code/matrix_multiplication_2.cpp}

\begin{table}[H]
\centering
\begin{tabular}{|c|c|c|c|}
\hline
\multirow{2}{*}{Rozmiar macierzy} & \multicolumn{3}{c|}{Rozmiar bloku} \\ \cline{2-4}
& 8x8 & 16x16 & 22x22 \\ \hline
176x176 & 1.47 & 1.38 & 2.56 \\ \hline
352x352 & 12.69 & 11.68 & 21.46 \\ \hline
528x528 & 38.57 & 36.82 & 67.02 \\ \hline
\end{tabular}
\caption{Czas obliczeń [ms] -- wersja 2.}
\end{table}

\newpage
\subsection{Wersja 3}

\subsubsection{Opis rozwiązania}

W trzecim podejściu wykorzystana została pamięć współdzielona. W kolejnych iteracjach pętli po blokach najpierw wczytywany jest blok do pamięci współdzielonej (każdy wątek wczytuje jedną komórkę), a następnie wykonywane są obliczenia na dostępnych danych. W tym podejściu niezbędne jest synchronizowanie się wątków dwukrotnie w każdej iteracji.

\lstinputlisting[caption=Mnożenie macierzy kwadratowych na GPU -- wersja 3.]{./code/matrix_multiplication_3.cpp}

\subsubsection{Teoretyczna zajętość SM}

\begin{center}
\begin{table}[H]
\centering
\begin{tabular}{|c|c|c|c|c|c|}
\hline
\multicolumn{2}{|c|}{\multirow{2}{*}{Kryterium}} & \multicolumn{3}{c|}{Teoretyczna wartość} & \multirow{2}{*}{Limit GPU} \\ \cline{3-5}
\multicolumn{2}{|c|}{} & 8x8 & 16x16 & 22x22 & \\ \hline
\multirow{4}{*}{Zajętość SM} & Aktywne bloki & 8 & 4 & 2 & 8 \\ \cline{2-6}
& Aktywne warpy & 16 & 32 & 32 & 32 \\ \cline{2-6}
& Aktywne wątki & 512 & 1024 & 968 & 1024 \\ \cline{2-6}
& Zajętość & 50\% & 100\% & 100\% & 100\% \\ \hline
\multirow{3}{*}{Warpy} & Wątki/Blok & 64 & 256 & 484 & 512 \\ \cline{2-6}
& Warpy/Blok & 2 & 8 & 16 & 16 \\ \cline{2-6}
& Limit bloków & 16 & \textcolor{red}{\textbf{4}} & \textcolor{red}{\textbf{2}} & 8 \\ \hline
\multirow{3}{*}{Rejestry} & Rejestry/Wątek & 12 & 12 & 13 & 128 \\ \cline{2-6}
& Rejestry/Blok & 1024 & 3072 & 6656 & 16384 \\ \cline{2-6}
& Limit bloków & 16 & 5 & \textcolor{red}{\textbf{2}} & 8 \\ \hline
\multirow{2}{*}{Pamięć współdzielona} & Pamięć współdzielona/Blok & 556 & 2092 & 3916 & 16384 \\ \cline{2-6}
& Limit bloków & 16 & 6 & 4 & 8 \\ \hline
\end{tabular}
\caption{Teoretyczna zajętość SM -- wersja 3.}
\end{table}
\end{center}

Podobnie jak dla 1. i 2. wersji, dla bloku 8x8 limitem okazuje się być maksymalna ilość bloków na SM, stąd zajętość $ 16 / 32 = 50\% $. \\
Dla macierzy 16x16 limitem są warpy. Przypada odpowiednio $ 8 $ warpów na blok, co daje limit $ 4 $ i $ 2 $ aktywnych bloków. Zajętość dla tej wielkości bloku wynosi $ 100\% $. \\
Dla macierzy 22x22 limitem są zarówno warpy jak i rejestry. Przypada $ 16 $ warpów na blok, co daje limit $ 2 $ aktywnych bloków. $ 6656 $ rejestrów na blok również daje limit $ 2 $ aktywnych bloków. Zajętość dla tej wielkości bloku wynosi $ 100\% $.

\subsubsection{Wyniki pomiarów}

\begin{enumerate}

\item \textbf{Czas trwania obliczeń} \newline

\begin{table}[H]
\centering
\begin{tabular}{|c|c|c|c|}
\hline
\multirow{2}{*}{Rozmiar macierzy} & \multicolumn{3}{c|}{Rozmiar bloku} \\ \cline{2-4}
& 8x8 & 16x16 & 22x22 \\ \hline
128x128 & 0.140 & 0.093 & \textcolor{gray}{n/d} \\ \hline
176x176 & 0.316 & 0.181 & 0.374 \\ \hline
256x256 & 0.930 & 0.519 & \textcolor{gray}{n/d} \\ \hline
352x352 & 2.351 & 1.297 & 2.343 \\ \hline
384x384 & 3.034 & 1.642 & \textcolor{gray}{n/d} \\ \hline
512x512 & 7.001 & 3.764 & \textcolor{gray}{n/d} \\ \hline
528x528 & 7.666 & 4.075 & 7.389 \\ \hline
640x640 & 13.820 & 7.301 & \textcolor{gray}{n/d} \\ \hline
\end{tabular}
\caption{Czas obliczeń [ms] -- wersja 3.}
\end{table}

\begin{figure}[H]
\centering
  \begin{tikzpicture}
    \begin{axis}[
      xlabel=Szerokość macierzy,
      ylabel={Czas wykonania [ms]},
      legend pos=north west,
      grid=both
    ]

    \addplot[color=blue,mark=square*] coordinates {%
      (128, 0.140)
      (176, 0.316)
      (256, 0.930)
      (352, 2.351)
      (384, 3.034)
      (512, 7.001)
      (528, 7.666)
      (640, 13.820)
    };
    \addlegendentry{8x8}

    \addplot[color=green,mark=triangle*] coordinates {%
      (128, 0.093)
      (176, 0.181)
      (256, 0.519)
      (352, 1.297)
      (384, 1.642)
      (512, 3.764)
      (528, 4.075)
      (640, 7.301)
    };
    \addlegendentry{16x16}

    \addplot[color=red,mark=*] coordinates {%
      (176, 0.374)
      (352, 2.343)
      (528, 7.389)
    };
    \addlegendentry{22x22}

    \end{axis}%
  \end{tikzpicture}%
\caption{Zależność pomiędzy czasem obliczeń a rozmiarem macierzy -- wersja 3.}
\end{figure}

\item \textbf{Ilość operacji zmiennoprzecinkowych na sekundę} \newline

\begin{table}[H]
\centering
\begin{tabular}{|c|c|c|c|}
\hline
\multirow{2}{*}{Rozmiar macierzy} & \multicolumn{3}{c|}{Rozmiar bloku} \\ \cline{2-4}
& 8x8 & 16x16 & 22x22 \\ \hline
128x128 & 29.857 & 45.291 & \textcolor{gray}{n/d}\\ \hline
176x176 & 34.491 & 60.169 & 29.138 \\ \hline
256x256 & 36.086 & 64.603 & \textcolor{gray}{n/d} \\ \hline
352x352 & 37.106 & 67.273 & 37.231 \\ \hline
384x384 & 37.322 & 68.970 & \textcolor{gray}{n/d} \\ \hline
512x512 & 38.341 & 71.314 & \textcolor{gray}{n/d} \\ \hline
528x528 & 38.402 & 72.246 & 39.844 \\ \hline
640x640 & 37.938 & 71.812 & \textcolor{gray}{n/d} \\ \hline
\end{tabular}
\caption{Ilosc operacji zmiennoprzecinkowych na sekundę (GFLOPS) -- wersja 3.}
\end{table}

\begin{figure}[H]
\centering
  \begin{tikzpicture}
    \begin{axis}[
      xlabel=Szerokość macierzy,
      ylabel={GFLOPS},
      legend pos=north west,
      grid=both
    ]

    \addplot[color=blue,mark=square*] coordinates {%
      (128, 29.857)
      (176, 34.491)
      (256, 36.086)
      (352, 37.106)
      (384, 37.322)
      (512, 38.341)
      (528, 38.402)
      (640, 37.938)
    };
    \addlegendentry{8x8}

    \addplot[color=green,mark=triangle*] coordinates {%
      (128, 45.291)
      (176, 60.169)
      (256, 64.603)
      (352, 67.273)
      (384, 68.970)
      (512, 71.314)
      (528, 72.246)
      (640, 71.812)
    };
    \addlegendentry{16x16}

    \addplot[color=red,mark=*] coordinates {%
      (176, 29.138)
      (352, 37.231)
      (528, 39.844)
    };
    \addlegendentry{22x22}

    \end{axis}%
  \end{tikzpicture}%
\caption{Zależność pomiędzy ilością operacji zmiennoprzecinkowychna sekundę a rozmiarem macierzy -- wersja 3.}
\end{figure}

\item \textbf{Ilość instrukcji wykonanych na sekundę} \newline

\begin{table}[H]
\centering
\begin{tabular}{|c|c|c|c|}
\hline
\multirow{2}{*}{Rozmiar macierzy} & \multicolumn{3}{c|}{Rozmiar bloku} \\ \cline{2-4}
& 8x8 & 16x16 & 22x22 \\ \hline
128x128 & 0.16386 & 0.15619 & \textcolor{gray}{n/d}\\ \hline
176x176 & 0.16910 & 0.23896 & 0.12664 \\ \hline
256x256 & 0.17538 & 0.26221 & \textcolor{gray}{n/d} \\ \hline
352x352 & 0.17807 & 0.28408 & 0.15949 \\ \hline
384x384 & 0.17667 & 0.29259 & \textcolor{gray}{n/d} \\ \hline
512x512 & 0.18015 & 0.29476 & \textcolor{gray}{n/d} \\ \hline
528x528 & 0.18043 & 0.30390 & 0.17317 \\ \hline
640x640 & 0.17746 & 0.29891 & \textcolor{gray}{n/d} \\ \hline
\end{tabular}
\caption{Ilość instrukcji wykonana na sekundę (GIPS) -- wersja 3.}
\end{table}

\begin{figure}[H]
\centering
  \begin{tikzpicture}
    \begin{axis}[
      xlabel=Szerokość macierzy,
      ylabel={GIPS},
      legend pos=north west,
      grid=both
    ]

    \addplot[color=blue,mark=square*] coordinates {%
      (128, 0.16386)
      (176, 0.16910)
      (256, 0.17538)
      (352, 0.17807)
      (384, 0.17667)
      (512, 0.18015)
      (528, 0.18043)
      (640, 0.17746)
    };
    \addlegendentry{8x8}

    \addplot[color=green,mark=triangle*] coordinates {%
      (128, 0.15619)
      (176, 0.23896)
      (256, 0.26221)
      (352, 0.28408)
      (384, 0.29259)
      (512, 0.29476)
      (528, 0.30390)
      (640, 0.29891)
    };
    \addlegendentry{16x16}

    \addplot[color=red,mark=*] coordinates {%
      (176, 0.12664)
      (352, 0.15949)
      (528, 0.17317)
    };
    \addlegendentry{22x22}

    \end{axis}%
  \end{tikzpicture}%
\caption{Zależność pomiędzy ilością instrukcji wykonanych na sekundę a rozmiarem macierzy -- wersja 3.}
\end{figure}

\item \textbf{CGMA} \newline

\begin{table}[H]
\centering
\begin{tabular}{|c|c|c|c|}
\hline
\multirow{2}{*}{Rozmiar macierzy} & \multicolumn{3}{c|}{Rozmiar bloku} \\ \cline{2-4}
& 8x8 & 16x16 & 22x22 \\ \hline
128x128 & 120.471 & 546.133 & \textcolor{gray}{n/d} \\ \hline
176x176 & 129.067 & 516.267 & 354.253 \\ \hline
256x256 & 128.502 & 520.127 & \textcolor{gray}{n/d} \\ \hline
352x352 & 128.000 & 507.803 & 382.302 \\ \hline
384x384 & 128.000 & 515.580 & \textcolor{gray}{n/d} \\ \hline
512x512 & 127.626 & 514.008 & \textcolor{gray}{n/d} \\ \hline
528x528 & 127.765 & 510.593 & 380.668 \\ \hline
640x640 & 127.760 & 510.723 & \textcolor{gray}{n/d} \\ \hline
\end{tabular}
\caption{Stosunek ilości operacji zmiennoprzecinkowych do ilości operacji odczytu/zapisu z pamięci globalnej -- wersja 3.}
\end{table}

\begin{figure}[H]
\centering
  \begin{tikzpicture}
    \begin{axis}[
      xlabel=Szerokość macierzy,
      ylabel={CGMA},
      legend pos=north west,
      grid=both
    ]

    \addplot[color=blue,mark=square*] coordinates {%
      (128, 120.471)
      (176, 129.067)
      (256, 128.502)
      (352, 128.000)
      (384, 128.000)
      (512, 127.626)
      (528, 127.765)
      (640, 127.760)
    };
    \addlegendentry{8x8}

    \addplot[color=green,mark=triangle*] coordinates {%
      (128, 546.133)
      (176, 516.267)
      (256, 520.127)
      (352, 507.803)
      (384, 515.580)
      (512, 514.008)
      (528, 510.593)
      (640, 510.723)
    };
    \addlegendentry{16x16}

    \addplot[color=red,mark=*] coordinates {%
      (176, 354.253)
      (352, 382.302)
      (528, 380.668)
    };
    \addlegendentry{22x22}

    \end{axis}%
  \end{tikzpicture}%
\caption{Zależność CGMA od rozmiaru macierzy -- wersja 3.}
\end{figure}

\end{enumerate}


\subsection{Wersja 4}

Jest to rozszerzona wersja 3 o równoległe z obliczeniami pobranie kolejnych danych (na poziomie bloku). Ma to spowodować złagodzenie kosztów synchronizacji.

\lstinputlisting[caption=Mnożenie macierzy kwadratowych na GPU -- wersja 4.]{./code/matrix_multiplication_4.cpp}

\begin{table}[H]
\centering
\begin{tabular}{|c|c|c|c|}
\hline
Rozmiar bloku & 16x16 \\ \hline
Macierz 64x64 & 0.02 \\ \hline
Macierz 128x128 & 0.10 \\ \hline
Macierz 256x256 & 0.60 \\ \hline
Macierz 384x384 & 1.89 \\ \hline
Macierz 512x512 & 4.44 \\ \hline
\end{tabular}
\caption{Czas obliczeń [ms] -- wersja 4.}
\end{table}


\subsection{Wersja 5}

Ostatnia wersja rozszerza wersję 4 -- każdy wątek wykonuje większą pracę. Zostało to zrealizowane przez zwiększenie pobieranych i obliczanych danych z jednej do czterech.

\lstinputlisting[caption=Mnożenie macierzy kwadratowych na GPU -- wersja 5.]{./code/matrix_multiplication_5.cpp}

\begin{table}[H]
\centering
\begin{tabular}{|c|c|c|}
\hline
\multirow{2}{*}{Rozmiar macierzy} & \multicolumn{2}{c|}{Rozmiar bloku} \\ \cline{2-3}
& 8x8 & 16x16 \\ \hline
128x128 & 0.118 & 0.139 \\ \hline
256x256 & 0.877 & 0.763 \\ \hline
384x384 & 2.622 & 2.287 \\ \hline
512x512 & 6.248 & 5.471 \\ \hline
640x640 & 12.047 & 10.632 \\ \hline
\end{tabular}
\caption{Czas obliczeń [ms] -- wersja 3.}
\end{table}

\subsubsection{Ilosc operacji zmiennoprzecinkowych na sekundę}

\begin{table}[H]
\centering
\begin{tabular}{|c|c|c|}
\hline
\multirow{2}{*}{Rozmiar macierzy} & \multicolumn{2}{c|}{Rozmiar bloku} \\ \cline{2-3}
& 8x8 & 16x16 \\ \hline
128x128 & 35.396 & 30.229 \\ \hline
256x256 & 38.248 & 43.995 \\ \hline
384x384 & 43.189 & 49.512 \\ \hline
512x512 & 42.966 & 49.063 \\ \hline
640x640 & 43.520 & 49.314 \\ \hline
\end{tabular}
\caption{Ilosc operacji zmiennoprzecinkowych na sekundę (GFLOPS) -- wersja 1.}
\end{table}

\subsubsection{Ilość instrukcji wykonana na sekundę}

\begin{table}[H]
\centering
\begin{tabular}{|c|c|c|c|}
\hline
\multirow{2}{*}{Rozmiar macierzy} & \multicolumn{2}{c|}{Rozmiar bloku} \\ \cline{2-3}
& 8x8 & 16x16 \\ \hline
128x128 & 0.1481 & 0.1856 \\ \hline
256x256 & 0.1423 & 0.1641 \\ \hline
384x384 & 0.1543 & 0.1951 \\ \hline
512x512 & 0.1574 & 0.1894 \\ \hline
640x640 & 0.1538 & 0.1909 \\ \hline
\end{tabular}
\caption{Ilość instrukcji wykonana na sekundę (GIPS) -- wersja 1.}
\end{table}

\subsubsection{CGMA}

\begin{table}[H]
\centering
\begin{tabular}{|c|c|c|c|}
\hline
\multirow{2}{*}{Rozmiar macierzy} & \multicolumn{2}{c|}{Rozmiar bloku} \\ \cline{2-3}
& 8x8 & 16x16 \\ \hline
128x128 & 240.941 & 1129.931 \\ \hline
256x256 & 256.250 & 1236.528 \\ \hline
384x384 & 250.776 & 1276.675 \\ \hline
512x512 & 253.050 & 1297.743 \\ \hline
640x640 & 255.393 & 1310.720 \\ \hline
\end{tabular}
\caption{Stosunek ilości operacji zmiennoprzecinkowych do ilości operacji odczytu/zapisu z pamięci globalnej -- wersja 1.}
\end{table}



\newpage

\newpage
\section{Podsumowanie}
Zdecydowaliśmy się na porównanie wyników dla największych badanych macierzy (o wymiarach 528x528 dla pierwszej grupy i 640x640 dla drugiej grupy) i bloków o wymiarach 8x8 i 16x16. W przypadku wersji 4. wybraliśmy tę z pobraniem do rejestru, ze względu na lepsze czasy wykonania.

\begin{table}[H]
\centering
\begin{tabular}{|c|c|c|c|c|c|}
\hline
Rozwiązanie & Rozmiar bloku & Czas & GFLOPS & GIPS & CGMA \\ \hline
\multirow{2}{*}{Wersja 1} & 8x8 & 1345.355 & 0.219 & 0.03774 & 5.333 \\ \cline{2-6}
& 16x16 & 349.936 & 0.841 & 0.14509 & 8.000 \\ \hline
\multirow{2}{*}{Wersja 2} & 8x8 & 38.574 & 7.632 & 0.08095 & 21.178 \\ \cline{2-6}
& 16x16 & 36.721 & 8.017 & 0.08248 & 32.267 \\ \hline
\multirow{2}{*}{Wersja 3} & 8x8 & 7.666 & 38.402 & 0.18043 & 127.765 \\ \cline{2-6}
& 16x16 & 4.075 & 72.246 & 0.30390 & 510.593 \\ \hline
\end{tabular}
\caption{Porównanie pierwszej grupy rozwiązań -- macierz 528x528.}
\end{table}

\begin{figure}[H]

\begin{minipage}[c]{0.46\textwidth}
\centering

\resizebox{\textwidth}{!}{%
\begin{tikzpicture}
\begin{axis}[
  ybar,%=5pt,
  ylabel=Czas wykonania,
  ymode=log,
  log origin=infty,
  enlargelimits=0.15,
  xtick=data,
  symbolic x coords={Wersja 1, Wersja 2, Wersja 3}
]
\addplot 
  coordinates {%
    (Wersja 1, 1345.355)
    (Wersja 2, 38.574)
    (Wersja 3, 7.666)
  };

\addplot 
  coordinates {%
    (Wersja 1, 349.936)
    (Wersja 2, 36.721) 
    (Wersja 3, 4.075)
  };

\end{axis}
\end{tikzpicture}
}

\vspace{18pt}

\resizebox{\textwidth}{!}{%
\begin{tikzpicture}
\begin{axis}[
  ybar,%=5pt,
  ylabel=GIPS,
  %ymode=log,
  log origin=infty,
  enlargelimits=0.15,
  xtick=data,
  symbolic x coords={Wersja 1, Wersja 2, Wersja 3}
]
\addplot 
  coordinates {%
    (Wersja 1, 0.03774)
    (Wersja 2, 0.08095)
    (Wersja 3, 0.18043)
  };

\addplot 
  coordinates {%
    (Wersja 1, 0.14509)
    (Wersja 2, 0.08248) 
    (Wersja 3, 0.30390)
  };

\end{axis}
\end{tikzpicture}
}
\end{minipage}
\qquad
\begin{minipage}[c]{0.46\textwidth}
\centering

\resizebox{\textwidth}{!}{%
\begin{tikzpicture}
\begin{axis}[
  ybar,%=5pt,
  ylabel=GFLOPS,
  ymode=log,
  log origin=infty,
  enlargelimits=0.15,
  xtick=data,
  symbolic x coords={Wersja 1, Wersja 2, Wersja 3}
]
\addplot 
  coordinates {%
    (Wersja 1, 0.219)
    (Wersja 2, 7.632)
    (Wersja 3, 38.402)
  };

\addplot 
  coordinates {%
    (Wersja 1, 0.841)
    (Wersja 2, 8.017) 
    (Wersja 3, 72.246)
  };

\end{axis}
\end{tikzpicture}
}

\vspace{18pt}

\resizebox{\textwidth}{!}{%
\begin{tikzpicture}
\begin{axis}[
  ybar,%=5pt,
  ylabel=CGMA,
  ymode=log,
  log origin=infty,
  enlargelimits=0.15,
  xtick=data,
  symbolic x coords={Wersja 1, Wersja 2, Wersja 3}
]
\addplot 
  coordinates {%
    (Wersja 1, 5.333)
    (Wersja 2, 21.178)
    (Wersja 3, 127.765)
  };

\addplot 
  coordinates {%
    (Wersja 1, 8.000)
    (Wersja 2, 32.267) 
    (Wersja 3, 510.593)
  };

\end{axis}
\end{tikzpicture}
}
\end{minipage}

\caption{Porównanie pierwszej grupy rozwiązań -- macierz 528x528.}

\end{figure}

\begin{description}

\item[Wersja 1.] charakteryzuje się zdecydowanie najdłuższym czasem obliczeń. Wykorzystanie jednego bloku w znacznym stopniu ogranicza równoległość przetwarzania, szczególnie dla bloków o małym rozmiarze. Przetwarzanie jest w tym przypadku ograniczone dostępem do pamięci -- obniżona jest ilość operacji zmiennoprzecinkowych, które mogłyby być wykonane w danym czasie.

\item[Wersja 2.] jest bardziej wydajna dzięki równoległości na poziomie między blokami. Ciekawy jest przypadek bloku 22x22, który daje czas przetwarzań porównywalny z czasem przetwarzania dla bloku 8x8, a jest gorszy od czasu przetwarzania dla bloku 16x16. Podejrzewamy, że wynika to z gorszej zajętości SM wątkami.

\item[Wersja 3.] jest najlepszą wersją kodu. Czas dostępu do pamięci współdzielonej jest setki razy szybszy niż czas dostępu do pamięci globalnej. Ponadto wątki równolegle ładują potrzebne do pamięci współdzielonej.

\end{description}

\begin{table}[H]
\centering
\begin{tabular}{|c|c|c|c|c|c|}
\hline
Rozwiązanie & Rozmiar bloku & Czas & GFLOPS & GIPS & CGMA \\ \hline
\multirow{2}{*}{Wersja 3} & 8x8 & 13.820 & 37.938 & 0.1775 & 127.760 \\ \cline{2-6}
& 16x16 & 7.301 & 71.812 & 0.2989 & 510.723 \\ \hline
\multirow{2}{*}{Wersja 4} & 8x8 & 13.950 & 37.583 & 0.1815 & 126.499 \\ \cline{2-6}
& 16x16 & 7.356 & 71.269 & 0.3013 & 500.764 \\ \hline
\multirow{2}{*}{Wersja 5} & 8x8 & 12.047 & 43.520 & 0.1538 & 255.393 \\ \cline{2-6}
& 16x16 & 10.632 & 49.314 & 0.1909 & 1310.720 \\ \hline
\end{tabular}
\caption{Porównanie drugiej grupy rozwiązań -- macierz 640x640.}
\end{table}

\begin{figure}[H]

\begin{minipage}[c]{0.46\textwidth}
\centering

\resizebox{\textwidth}{!}{%
\begin{tikzpicture}
\begin{axis}[
  ybar,%=5pt,
  ylabel=Czas wykonania,
  %ymode=log,
  log origin=infty,
  enlargelimits=0.15,
  xtick=data,
  symbolic x coords={Wersja 3, Wersja 4, Wersja 5}
]
\addplot 
  coordinates {%
    (Wersja 3, 13.820)
    (Wersja 4, 13.950)
    (Wersja 5, 12.047)
  };

\addplot 
  coordinates {%
    (Wersja 3, 7.301)
    (Wersja 4, 7.356) 
    (Wersja 5, 10.632)
  };

\end{axis}
\end{tikzpicture}
}

\vspace{18pt}

\resizebox{\textwidth}{!}{%
\begin{tikzpicture}
\begin{axis}[
  ybar,%=5pt,
  ylabel=GIPS,
  %ymode=log,
  log origin=infty,
  enlargelimits=0.15,
  xtick=data,
  symbolic x coords={Wersja 3, Wersja 4, Wersja 5}
]
\addplot 
  coordinates {%
    (Wersja 3, 0.1775)
    (Wersja 4, 0.1815)
    (Wersja 5, 0.1538)
  };

\addplot 
  coordinates {%
    (Wersja 3, 0.2989)
    (Wersja 4, 0.3013) 
    (Wersja 5, 0.1909)
  };

\end{axis}
\end{tikzpicture}
}
\end{minipage}
\qquad
\begin{minipage}[c]{0.46\textwidth}
\centering

\resizebox{\textwidth}{!}{%
\begin{tikzpicture}
\begin{axis}[
  ybar,%=5pt,
  ylabel=GFLOPS,
  %ymode=log,
  log origin=infty,
  enlargelimits=0.15,
  xtick=data,
  symbolic x coords={Wersja 3, Wersja 4, Wersja 5}
]
\addplot 
  coordinates {%
    (Wersja 3, 37.938)
    (Wersja 4, 37.583)
    (Wersja 5, 43.520)
  };

\addplot 
  coordinates {%
    (Wersja 3, 71.812)
    (Wersja 4, 71.269) 
    (Wersja 5, 49.314)
  };

\end{axis}
\end{tikzpicture}
}

\vspace{18pt}

\resizebox{\textwidth}{!}{%
\begin{tikzpicture}
\begin{axis}[
  ybar,%=5pt,
  ylabel=CGMA,
  ymode=log,
  log origin=infty,
  enlargelimits=0.15,
  xtick=data,
  symbolic x coords={Wersja 3, Wersja 4, Wersja 5}
]
\addplot 
  coordinates {%
    (Wersja 3, 127.760)
    (Wersja 4, 126.499)
    (Wersja 5, 255.393)
  };

\addplot 
  coordinates {%
    (Wersja 3, 510.723)
    (Wersja 4, 500.764) 
    (Wersja 5, 1310.720)
  };

\end{axis}
\end{tikzpicture}
}
\end{minipage}

\caption{Porównanie drugiej grupy rozwiązań -- macierz 640x640.}

\end{figure}

\begin{description}

\item[Wersja 4.] nie spełniła naszych oczekiwań. Wprowadzone zmiany w porównaniu z wersją 3. nie przyniosły większych zmian, co więcej czas obliczeń w większości przypadków nieznacznie się pogorszył. W przypadku pobrania do pamięci współdzielonej spadek wydajności jest wiekszy, dla bloku o rozmiarze 16x16 ograniczeniem stała się pamięć współdzielona -- zajętość SM warpami spadła do $75\%$.

\item[Wersja 5.] w przypadku bloku o rozmiarze 8x8 spowodowała zauważalne poprawienie wydajności. Dla bloku 16x16 spadła zajętość -- zbyt duża liczba wykorzystanych rejestrów spowodowała spadek zajętości do $25\%$\footnote{Duża ilość rejestrów była spowodowana optymalizacjami kompilatora.}.

\end{description}


\section{Załączniki}

\begin{enumerate}
\item sekwencyjny1.lp -- plik z testem~\ref{test1}
\item sekwencyjny2.lp -- plik z testem~\ref{test2}
\item sekwencyjny3.lp -- plik z testem~\ref{test3}
\item rownolegly1.lp -- plik z testem~\ref{test4}
\item rownolegly2.lp -- plik z testem~\ref{test5}
\end{enumerate}

\vspace{1.4 cm}
W testach modelu sekwencyjnego wykorzystaliśmy 2 dodatkowe zmienne -- $y_{1}$ i $y_{2}$:
\begin{equation} \label{eq:y1y2}
\begin{array}{l}
y_{1} = 0 \land y_{2} = 1 \Rightarrow \text{pierwszy model sekwencyjny} \\
y_{1} = 1 \land y_{2} = 0 \Rightarrow \text{drugi model sekwencyjny} \\
y_{1} = 1 \land y_{2} = 1 \Rightarrow \text{trzeci model sekwencyjny}
\end{array}
\end{equation}

\newpage


\pagestyle{plain}
\newpage
\listoffigures
\addcontentsline{toc}{section}{Spis rysunków}
\newpage
\listoftables
\addcontentsline{toc}{section}{Spis tablic}
\newpage
\lstlistoflistings
\addcontentsline{toc}{section}{Spis kodów źródłowych}


\end{document}
