
\subsection{Wersja 1}

W pierwszej wersji programu wykorzystany został tylko jeden blok wątków. Jeśli rozmiar bloku jest równy rozmiarowi macierzy to każdy z wątków oblicza 1 element wyniku. W przeciwnym przypatku każdy z wątków oblicza $$ {\left(\frac{\text{rozmiar macierzy}}{\text{rozmiar bloku}}\right)}^{2} $$ elementów wyniku. Pamięć współdzielona nie jest wykorzystywana.

\lstinputlisting[caption=Mnożenie macierzy kwadratowych na GPU -- wersja 1.]{./code/matrix_multiplication_1.cpp}

Ponieważ wykorzystany jest tylko jeden blok, obliczenia przeprowadzane są na jednym SM, a co za tym idzie w danej chwili aktywny może być tylko jeden warp. \\
Duży wpływ na prędkość przetwarzania ma wielkość bloku, dla małych bloków efektywność jest bardzo niska -- duża ilość dostępów do pamięci ogranicza prędkość przetwarzania.

\begin{table}[H]
\centering
\begin{tabular}{|c|c|c|c|}
\hline
\multirow{2}{*}{Rozmiar macierzy} & \multicolumn{3}{c|}{Rozmiar bloku} \\ \cline{2-4}
& 8x8 & 16x16 & 22x22 \\ \hline
176x176 & 49.79 & 13.17 & 7.45 \\ \hline
352x352 & 401.35 & 103.85 & 59.70 \\ \hline
528x528 & 1354.40 & 349.77 & 200.36 \\ \hline
\end{tabular}
\caption{Czas obliczeń [ms] -- wersja 1.}
\end{table}

