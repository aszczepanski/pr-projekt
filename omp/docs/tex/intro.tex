\section{Wstęp}

\subsection{Opis problemu}

Głównym założeniem projektu było zapoznanie się biblioteką OpenMP na podstawie równoległego sumowania komórek tablicy. W ramach projektu zrealizowaliśmy 2 algorytmy sekwencyjne oraz 2 algorytmy zrównleglone. Celem dwóch różnych algorytmów sekwencyjnych było zbadanie wpływu sekcyjności pamięci podręcznej na czas realizacji zadania. W przypadku algorytmów zrównoleglonych badaliśmy wpływ kolejności uszeregowania pętli na końcowy rezultat.

\subsection{Zastosowane algorytmy}

\subsubsection{Podstawowy algorytm sekwencyjny}

\lstinputlisting{./code/sums.cpp}

\subsubsection{Algorytm sekwencyjny - sekcyjność pamięci}

\lstinputlisting{./code/sums2.cpp}

\subsubsection{Algorytm zrównoleglony - wpierw wiersze}

\lstinputlisting{./code/omp1.cpp}

\subsubsection{Algorytm zrównoleglony - wpierw kolumny}

\lstinputlisting{./code/omp2.cpp}