
\subsection{Wersja 5}

\subsubsection{Opis rozwiązania}

Ostatnia wersja rozszerza wersję 4 -- każdy wątek wykonuje większą pracę. Zostało to zrealizowane przez zwiększenie obliczanych danych z jednej do czterech.

\lstinputlisting[caption=Mnożenie macierzy kwadratowych na GPU -- wersja 5.]{./code/matrix_multiplication_5.cpp}

\subsubsection{Teoretyczna zajętość SM}

\begin{center}
\begin{table}[H]
\centering
\begin{tabular}{|c|c|c|c|c|}
\hline
\multicolumn{2}{|c|}{\multirow{2}{*}{Kryterium}} & \multicolumn{2}{c|}{Teoretyczna wartość} & \multirow{2}{*}{Limit GPU} \\ \cline{3-4}
\multicolumn{2}{|c|}{} & 8x8 & 16x16 & \\ \hline
\multirow{4}{*}{Zajętość SM} & Aktywne bloki & 8 & 1 & 8 \\ \cline{2-5}
& Aktywne warpy & 16 & 8 & 32 \\ \cline{2-5}
& Aktywne wątki & 512 & 256 & 1024 \\ \cline{2-5}
& Zajętość & 50\% & 25\% & 100\% \\ \hline
\multirow{3}{*}{Warpy} & Wątki/Blok & 64 & 256 & 512 \\ \cline{2-5}
& Warpy/Blok & 2 & 8 & 16 \\ \cline{2-5}
& Limit bloków & 16 & 4 & 8 \\ \hline
\multirow{3}{*}{Rejestry} & Rejestry/Wątek & 23 & 34 & 128 \\ \cline{2-5}
& Rejestry/Blok & 1536 & 8704 & 16384 \\ \cline{2-5}
& Limit bloków & 10 & \textcolor{red}{1} & 8 \\ \hline
\multirow{2}{*}{Pamięć współdzielona} & Pamięć współdzielona/Blok & 1068 & 4140 & 16384 \\ \cline{2-5}
& Limit bloków & 10 & 3 & 8 \\ \hline
\end{tabular}
\caption{Teoretyczna zajętość SM -- wersja 5.}
\end{table}
\end{center}

Podobnie jak dla poprzednich wersji, dla bloku 8x8 limitem okazuje się być maksymalna ilość bloków na SM, stąd zajętość $ 16 / 32 = 50\% $. \\
Dla macierzy 16x16 limitem są rejestry. $8704$ rejestrów na blok daje limit $1$ aktywnych bloków. Zajętość dla bloku 16x16 wynosi zaledwie $ 25\% $. \\

\subsubsection{Wyniki pomiarów}

\begin{enumerate}

\item \textbf{Czas trwania obliczeń} \newline

\begin{table}[H]
\centering
\begin{tabular}{|c|c|c|}
\hline
\multirow{2}{*}{Rozmiar macierzy} & \multicolumn{2}{c|}{Rozmiar bloku} \\ \cline{2-3}
& 8x8 & 16x16 \\ \hline
128x128 & 0.118 & 0.139 \\ \hline
256x256 & 0.877 & 0.763 \\ \hline
384x384 & 2.622 & 2.287 \\ \hline
512x512 & 6.248 & 5.471 \\ \hline
640x640 & 12.047 & 10.632 \\ \hline
\end{tabular}
\caption{Czas obliczeń [ms] -- wersja 5.}
\end{table}

\item \textbf{Ilość operacji zmiennoprzecinkowych na sekundę} \newline

\begin{table}[H]
\centering
\begin{tabular}{|c|c|c|}
\hline
\multirow{2}{*}{Rozmiar macierzy} & \multicolumn{2}{c|}{Rozmiar bloku} \\ \cline{2-3}
& 8x8 & 16x16 \\ \hline
128x128 & 35.396 & 30.229 \\ \hline
256x256 & 38.248 & 43.995 \\ \hline
384x384 & 43.189 & 49.512 \\ \hline
512x512 & 42.966 & 49.063 \\ \hline
640x640 & 43.520 & 49.314 \\ \hline
\end{tabular}
\caption{Ilosc operacji zmiennoprzecinkowych na sekundę (GFLOPS) -- wersja 5.}
\end{table}

\item \textbf{Ilość instrukcji na sekundę} \newline

\begin{table}[H]
\centering
\begin{tabular}{|c|c|c|c|}
\hline
\multirow{2}{*}{Rozmiar macierzy} & \multicolumn{2}{c|}{Rozmiar bloku} \\ \cline{2-3}
& 8x8 & 16x16 \\ \hline
128x128 & 0.1481 & 0.1856 \\ \hline
256x256 & 0.1423 & 0.1641 \\ \hline
384x384 & 0.1543 & 0.1951 \\ \hline
512x512 & 0.1574 & 0.1894 \\ \hline
640x640 & 0.1538 & 0.1909 \\ \hline
\end{tabular}
\caption{Ilość instrukcji wykonana na sekundę (GIPS) -- wersja 5.}
\end{table}

\item \textbf{CGMA} \newline

\begin{table}[H]
\centering
\begin{tabular}{|c|c|c|c|}
\hline
\multirow{2}{*}{Rozmiar macierzy} & \multicolumn{2}{c|}{Rozmiar bloku} \\ \cline{2-3}
& 8x8 & 16x16 \\ \hline
128x128 & 240.941 & 1129.931 \\ \hline
256x256 & 256.250 & 1236.528 \\ \hline
384x384 & 250.776 & 1276.675 \\ \hline
512x512 & 253.050 & 1297.743 \\ \hline
640x640 & 255.393 & 1310.720 \\ \hline
\end{tabular}
\caption{Stosunek ilości operacji zmiennoprzecinkowych do ilości operacji odczytu/zapisu z pamięci globalnej -- wersja 5.}
\end{table}

\end{enumerate}
