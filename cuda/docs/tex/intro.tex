\section{Wprowadzenie}

Celem projektu było zapoznanie się z możliwościami przetwarzania na kartach graficznych na przykładzie technologi CUDA.\\
Przygotowaliśmy 6 wersji programu, którego zadaniem było mnożenie macierzy kwadratowych na GPU:
\begin{enumerate}
\item wykorzystanie 1 bloku wątków
\item wykorzystanie gridu wieloblokowego wątków
\item wykorzystanie gridu wieloblokowego wątków i pamięci współdzielonej
\item wykorzystanie gridu wieloblokowego wątków i pamięci współdzielonej, zrównoleglone pobranie danych i obliczenia
\begin{enumerate}[(a)]
\item pobranie danych do rejestru
\item pobranie danych do pamięci współdzielonej
\end{enumerate}
\item wykorzystanie gridu wieloblokowego wątków i pamięci współdzielonej, zrównoleglone pobranie danych i obliczenia, powiększona ilość pracy każdego wątku
\end{enumerate}

Dla każdej wersji podajemy teoretyczną zajętość SM wynikającą z rozmiaru bloku, wykorzystanej liczby rejestrów, rozmiaru pamięci współdzielonej, a także z ograniczeń GPU\@. Wyjątkiem jest wersja 1. w której wykorzystany jest tylko 1 blok, zatem zajętość SM wynika tylko z jego rozmiaru.\\
W przypadku pozostałych wersji, dla poszczególnych parametrów podany jest w tabeli limit bloków -- oznacza on ile bloków maksymalnie można powiązać z SM przy danych parametrach. Jeśli wszystkie limity są większe od limitu bloków na SM wynikającego z ograniczeń GPU, to limit bloków GPU determinuje ilość aktywnych bloków, a co za tym idzie zajętość SM.\\
Efektywność programów zbadaliśmy także przy pomocy profilera NVIDIA Visual Profiler. Dla każdej instancji podajemy:
\begin{itemize}
\item czas wykonania
\item ilość operacji zmienno przecinkowych na sekundę (GFLOPS)
\item ilość instrukcji wykonanych na sekundę (GIPS)
\item stosunek operacji zmienno przecinkowych do ilości operacji odczytu/zapisu z pamięci globalnej (CGMA)
\end{itemize}

Ze względu na zupełnie odmienne podejścia w wersjach 1, 2 i 3, oraz modyfikacje wersji 3. w wersjach 4a, 4b i 5 rozmiary macierzy i bloków podzieliliśmy na dwie grupy:
\begin{itemize}
\item Wersje 1, 2 i 3 -- macierze 176x176, 352x352 oraz 528x528, bloki 8x8, 16x16, 22x22 (wymiary macierzy są podzielne przez 8, 16 i 22).
\item Wersje: 3, 4a, 4b, 5  -- macierze 128x128, 256x256, 384x384, 512x512, 640x640, bloki 8x8, 16x16 (wymiary macierzy podzielne przez 16 i 32).
\end{itemize}

Na zakończenie prezentujemy porównanie efektywności wszystkich badanych wersji.
