\section{Pomiary efektywności}

\subsection{Przyspieszenie obliczeń równoległych}

\subsubsection{Pomiary}

\begin{table}[H]
\centering
\begin{tabular}{|c|c|c|}
\hline
Algorytm & Czas wykonania w $ms$ & Przyspieszenie względem $sum\_ij$ \\ \hline
$sum\_ij$ & 253 & 1.000 \\ \hline
$sum\_ji$ & 2362 & 0.107 \\ \hline
$sum\_par\_ij$ & 87 & 2.908 \\ \hline
$sum\_par\_ji$ & 1189 & 0.213 \\ \hline
\end{tabular}
\caption{Porównanie szybkości wybranych algorytmów wobec $sum\_ij$}
\end{table}

Warto zwrócić uwagę na fakt, że oprócz braku lub obecności zrównoleglenia znaczący wpływ na szybkość przetwarzania ma kolejność uszeregowania pętli. Zostało to szerzej omówione w sekcji \ref{sec:trafienia}. Przyspieszenia dla poszczególnych uszeregowań pętli prezentują się następująco:

\begin{table}[H]
\centering
\begin{tabular}{|c|c|c|}
\hline
Algorytm & Czas wykonania w $ms$ & Przyspieszenie względem $sum\_ij$ \\ \hline
$sum\_ij$ & 2362 & 1.000 \\ \hline
$sum\_par\_ij$ & 87 & 2.908 \\ \hline
\end{tabular}
\caption{Porównanie szybkości wybranych algorytmów (o kolejności pętli $ij$) wobec $sum\_ij$}
\end{table}

\begin{table}[H]
\centering
\begin{tabular}{|c|c|c|}
\hline
Algorytm & Czas wykonania w $ms$ & Przyśpieszenie względem $sum\_ji$ \\ \hline
$sum\_ji$ & 2362 & 1.000 \\ \hline
$sum\_par\_ji$ & 1189 & 1.987 \\ \hline
\end{tabular}
\caption{Porównanie szybkości wybranych algorytmów (o kolejności pętli $ji$) wobec $sum\_ji$}
\end{table}

\subsubsection{Podsumowanie}

Zrównoleglenie przetwarzania znacząco przyśpiesza jego czas przetwarzania. Znaczny wpływ na wielkość przyspieszenia ma kolejność uszeregowania pętli. W przybliżeniu jest to:
\begin{itemize}
\item trzykrotne przyśpieszenie dla uszeregowania pętli $ij$
\item dwukrotne przyśpieszenie dla uszeregowania pętli $ji$
\end{itemize}

\subsection{Ilość wykonanych instrukcji na jeden cykl procesora}

IPC (insructions per cycle) jest jednym z wyznaczników prędkości procesora. Oznacza on liczbę wykonywanych instrukcji przez procesor w jednym cyklu zegara. Wskaźnik IPC obliczaliśmy dla każdej funkcji na podstawie wzoru:
\begin{equation}
  IPC = \frac{ret\_instr}{CPU\_clocks}.
\end{equation}

CPI (cycles per instruction) jest odwrotnością IPC:
\begin{equation}
  CPI = \frac{CPU\_clocks}{ret\_instr}.
\end{equation}

\begin{table}[H]
\centering
\begin{tabular}{|c|c|c|c|c|}
\hline
Algorytm & $ret\_instr$ & $CPU\_clocks$ & $IPC$ & $CPI$ \\ \hline
$sum\_ij$ & 12640 & 14736 & 0.86 & 1.17 \\ \hline
$sum\_ji$ & 9780 & 136048 & 0.07 & 13.91 \\ \hline
$sum\_par\_ij$ & 12684 & 20096 & 0.63 & 1.58 \\ \hline
$sum\_par\_ji$ & 11524 & 259628 & 0.04 & 22.53 \\ \hline
$sum\_sec$ & 14420 & 1139472 & 0.01 & 79.02 \\ \hline
$sum\_pf$ & 14996 & 13140 & 1.14 & 0.88 \\ \hline
\end{tabular}
\caption{Wartości IPC i CPI dla poszczególnych algorytmów}
\end{table}

Zgodnie z oczekiwaniami najwyższe wartości wzkaźnika IPC (a zarazem najniższe CPI) wysętpują w przypadku algorytmów, które spełniają zasadę lokalności przestrzennej, są to:
\begin{itemize}
\item $sum\_ij$
\item $sum\_par\_ij$
\item $sum\_pf$
\end{itemize}

IPC dla wyżej wymienionej grupy algorytmów jest kilkunasto, a w niektórych przypadkach kilkudziesiącio krotonie większe niż dla algorytmów z pozostałej grupy.

\subsection{Współczynniki braku trafień do pamięci}
\label{sec:trafienia}

\subsubsection{Znaczenie danych}

\begin{table}[H]
\centering
\begin{tabular}{|l|l|}
\hline
Nazwa & Znaczenie \\ \hline
DC acceesses & ilość odwołań do pamięci podręcznej \\ \hline
DC misses & ilość chybień do pamięci podręcznej \\ \hline
L2 requests & ilość odwołań do pamięci L2 \\ \hline
L2 misses & ilość chybień do pamięci L2 \\ \hline
L2 fill write & zapis danych w L2 powodowany usunięciem ich z pp L1 \\ \hline
DRAM accesses & ilość odwołań do pamięci dynamicznej \\ \hline
\end{tabular}
\caption{Badane dane}
\end{table}


\subsubsection{Pomiary}

Wartości $DC\_accesss$ oraz $DC\_missess$ odczytaliśmy bezpośrednio z programu CodeAnalyst. Natomiast $DC\_miss\_ratio$ oraz $DC\_miss\_rate$ obliczyliśmy w następujący sposób:

\begin{equation}
  DC\_miss\_rate = \frac{DC\_missess}{ret\_instr}
\end{equation}

\begin{equation}
  DC\_miss\_ratio = \frac{DC\_missess}{DC\_accesss}
\end{equation}

\begin{table}[H]
\centering
\begin{tabular}{|l|c|c|c|c|}
\hline
Algorytm & DC accessess & DC missess & DC miss ratio & DC miss rate \\ \hline
$sum\_ij$ & 5792 & 24 & $0\%$ & $0\%$ \\ \hline
$sum\_ji$ & 7620 & 4068 & $53\%$ & $42\%$ \\ \hline
$sum\_par\_ij$ & 5188 & 20 & $0\%$ & $0\%$ \\ \hline
$sum\_par\_ji$ & 7652 & 4456 & $58\%$ & $38\%$ \\ \hline
$sum\_sec$ & 18808 & 5256 & $28\%$ & $36\%$ \\ \hline
$sum\_pf$ & 9928 & 24 & $0\%$ & $0\%$ \\ \hline
\end{tabular}
\caption{Pomiary związane z odwołaniami do pamięci L1}
\end{table}

Podobnie jak w przypadku pamięci L1 wartości $L2\_requests$, $L2\_misses$ oraz $L2\_fill\_write$ odczytaliśmy bezpośrednio z programu CodeAnalyst. Natomiast $L2\_miss\_ratio$ oraz $L2\_miss\_rate$ obliczyliśmy w następujący sposób:

\begin{equation}
  L2\_miss\_rate = \frac{L2\_missess}{ret\_instr}
\end{equation}

\begin{equation}
  L2\_miss\_ratio = \frac{L2\_missess}{L2\_requests + L2\_fill\_write}
\end{equation}

\begin{table}[H]
\centering
\begin{tabular}{|l|c|c|c|c|c|}
\hline
Algorytm & L2 requests & L2 missess & L2 fill write & L2 miss rate & L2 miss ratio \\ \hline
$sum\_ij$ & 663 & 340 & 684 & $3\%$ & $25\%$ \\ \hline
$sum\_ji$ & 5452 & 7209 & 10908 & $56\%$ & $30\%$ \\ \hline
$sum\_par\_ij$ & 1431 & 352 & 668 & $3\%$ & $17\%$ \\ \hline
$sum\_par\_ji$ & 5472 & 23736 & 10844 & $47\%$ & $16\%$ \\ \hline
$sum\_sec$ & 11024 & 15945 & 21668 & $76\%$ & $29\%$ \\ \hline
$sum\_pf$ & 344 & 693 & 684 & $2\%$ & $25\%$ \\ \hline
\end{tabular}
\caption{Pomiary związane z odwołaniami do pamięci L2}
\end{table}

\subsubsection{Omówienie}

Znowu można wyróżnić 2 grupy algorytmów. Pierwsza z nich składająca się z:

\begin{itemize}
\item $sum\_ij$
\item $sum\_par\_ij$
\item $sum\_pf$
\end{itemize}

cechują się one lokalnością przestrzenną. Widać to bardzo wyraźnie przy współczynnikach $DC\_miss\_ratio$ oraz $DC\_miss\_rate$. Dla wyżej wymienionej grupy wynoszą one w przybliżeniu $0\%$, a dla pozostałych algorytmów tj $sum\_ji$, $sum\_par\_ji$ oraz $sum\_sec$ po kilkadziesiąt procent. Związane to jest z faktem że grupa posiadająca cechę lokalności przestrzennej sumuje wierszami, zatem z raz wczytanej linii do pamięci podręcznej wszystkie elementy są wykorzystywane jeden po drugim.
