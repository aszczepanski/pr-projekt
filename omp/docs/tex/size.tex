\section{Wpływ rozmiaru danych}

\subsection{Wstęp}

W celu sprawdzenia wpływu rozmiaru danych na czas realizacji zadania dokonaliśmy pomiarów czasów dla dwóch rozmiarów danych:

\begin{itemize}
\item Instancja A: $tab[2^{28}][2^4]$
\item Instancja B: $tab[2^{24}][2^4]$
\end{itemize}

Spodziewaliśmy się, że czas obliczeń dla Instancji B będzie kilkukrotnie krótszy.

\subsection{Wyniki}

Wyniki pomiarów prezentują się następująco:

\begin{table}[!ht]
\caption{Czas realizacji kodu dla poszczególnych funkcji w $ms$.}
\begin{tabular}{|c|c|c|c|c|c|c|}

\hline
  Problem &
  $sum\_ij$ &
  $sum\_ji$ &
  $sum\_sec$ &
  $sum\_pf$ &
  $sum\_par\_ij$ &
  $sum\_par\_ji$ \\
\hline Instancja A &
  253 &
  2352 &
  19552 &
  225 &
  87 &
  1189 \\
\hline Instancja B &
  62 &
  59 &
  4890 &
  57 &
  23 &
  304 \\
\hline Stosunek A/B &
  4.081 &
  4.003 &
  3.998 &
  3.947 &
  3.783 &
  3.911 \\
\hline

\end{tabular}
\end{table}

\subsection{Podsumowanie}

Wyniki eksperymentu są bardzo zadowalające. W przypadku każdego algorytmu rozmiar danych ma liniowy wpływ na czas przetwarzania. Jest to zgodne ze złożonością algorytmu która jest liniowa wobec $n$ gdzie $n$ to $liczba\_wierszy * liczba\_kolumn$.\newline

W każdym przypadku czas realizacji dla Instancji B był około $4$ razy krótszy niż dla Instancji A. Średnia tych wartości wynosi $3.954$.
