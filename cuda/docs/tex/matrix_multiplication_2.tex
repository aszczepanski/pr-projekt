
\subsection{Wersja 2}

\subsubsection{Opis rozwiązania}

W drugiej wersji wykorzystywany jest grid wieloblokowy o rozmiarze\footnotemark $$\frac{\text{szerokość macierzy}}{\text{szerokość bloku}} \times \frac{\text{wysokość macierzy}}{\text{wysokość bloku}}$$
\footnotetext{W analizowanych przypadkach wysokość macierzy jest równa szerokości macierzy, wysokość bloku jest równa szerokości bloku.}
Każdy wątek oblicza jeden element macierzy wynikowej. Pamięć współdzielona nie jest wykorzystywana.

\lstinputlisting[caption=Mnożenie macierzy kwadratowych na GPU -- wersja 2.]{./code/matrix_multiplication_2.cpp}

\subsubsection{Teoretyczna zajętość SM}

\begin{center}
\begin{table}[H]
\centering
\begin{tabular}{|c|c|c|c|c|c|}
\hline
\multicolumn{2}{|c|}{\multirow{2}{*}{Kryterium}} & \multicolumn{3}{c|}{Teoretyczna wartość} & \multirow{2}{*}{Limit GPU} \\ \cline{3-5}
\multicolumn{2}{|c|}{} & 8x8 & 16x16 & 22x22 & \\ \hline
\multirow{4}{*}{Zajętość SM} & Aktywne bloki & 8 & 4 & 2 & 8 \\ \cline{2-6}
& Aktywne warpy & 16 & 32 & 32 & 32 \\ \cline{2-6}
& Aktywne wątki & 512 & 1024 & 968 & 1024 \\ \cline{2-6}
& Zajętość & 50\% & 100\% & 100\% & 100\% \\ \hline
\multirow{3}{*}{Warpy} & Wątki/Blok & 64 & 256 & 484 & 512 \\ \cline{2-6}
& Warpy/Blok & 2 & 8 & 16 & 16 \\ \cline{2-6}
& Limit bloków & 16 & \textcolor{red}{4} & \textcolor{red}{2} & 8 \\ \hline
\multirow{3}{*}{Rejestry} & Rejestry/Wątek & 10 & 10 & 10 & 128 \\ \cline{2-6}
& Rejestry/Blok & 1024 & 2560 & 5120 & 16384 \\ \cline{2-6}
& Limit bloków & 16 & 6 & 3 & 8 \\ \hline
\multirow{2}{*}{Pamięć współdzielona} & Pamięć współdzielona/Blok & 44 & 44 & 44 & 16384 \\ \cline{2-6}
& Limit bloków & 32 & 32 & 32 & 8 \\ \hline
\end{tabular}
\caption{Teoretyczna zajętość SM -- wersja 2.}
\end{table}
\end{center}

Podobnie jak w 1. wersji, dla bloku 8x8 limitem okazuje się być maksymalna ilość bloków na SM, stąd zajętość $ 16 / 32 = 50\% $. \\
Dla macierzy 16x16 i 22x22 limitem są warpy. Przypada odpowiednio $ 8 $ i $ 16 $ warpów na blok, co daje limit $ 4 $ i $ 2 $ aktywnych bloków. Zajętość dla obu tych wielkości bloków ponownie wynosi $ 100\% $.

\subsubsection{Wyniki pomiarów}

\begin{enumerate}

\item \textbf{Czas trwania obliczeń} \newline

\begin{table}[H]
\centering
\begin{tabular}{|c|c|c|c|}
\hline
\multirow{2}{*}{Rozmiar macierzy} & \multicolumn{3}{c|}{Rozmiar bloku} \\ \cline{2-4}
& 8x8 & 16x16 & 22x22 \\ \hline
176x176 & 1.466 & 1.381 & 2.552 \\ \hline
352x352 & 12.934 & 11.713 & 21.462 \\ \hline
528x528 & 38.574 & 36.721 & 66.967 \\ \hline
\end{tabular}
\caption{Czas obliczeń [ms] -- wersja 2.}
\end{table}

\item \textbf{Ilość operacji zmiennoprzecinkowych na sekundę} \newline

\begin{table}[H]
\centering
\begin{tabular}{|c|c|c|c|}
\hline
\multirow{2}{*}{Rozmiar macierzy} & \multicolumn{3}{c|}{Rozmiar bloku} \\ \cline{2-4}
& 8x8 & 16x16 & 22x22 \\ \hline
176x176 & 7.440 & 7.897 & 4.273 \\ \hline
352x352 & 6.744 & 7.447 & 4.064 \\ \hline
528x528 & 7.632 & 8.017 & 4.396 \\ \hline
\end{tabular}
\caption{Ilosc operacji zmiennoprzecinkowych na sekundę (GFLOPS) -- wersja 2.}
\end{table}

\item \textbf{Ilość instrukcji wykonanych na sekundę} \newline

\begin{table}[H]
\centering
\begin{tabular}{|c|c|c|c|}
\hline
\multirow{2}{*}{Rozmiar macierzy} & \multicolumn{3}{c|}{Rozmiar bloku} \\ \cline{2-4}
& 8x8 & 16x16 & 22x22 \\ \hline
176x176 & 0.07851 & 0.08334 & 0.04509 \\ \hline
352x352 & 0.07085 & 0.07776 & 0.04455 \\ \hline
528x528 & 0.08095 & 0.08248 & 0.04777 \\ \hline
\end{tabular}
\caption{Ilość instrukcji wykonana na sekundę (GIPS) -- wersja 2.}
\end{table}

\item \textbf{CGMA} \newline

\begin{table}[H]
\centering
\begin{tabular}{|c|c|c|c|}
\hline
\multirow{2}{*}{Rozmiar macierzy} & \multicolumn{3}{c|}{Rozmiar bloku} \\ \cline{2-4}
& 8x8 & 16x16 & 22x22 \\ \hline
176x176 & 21.511 & 35.852 & 23.178 \\ \hline
352x352 & 21.422 & 32.538 & 21.322 \\ \hline
528x528 & 21.178 & 32.267 & 21.768 \\ \hline
\end{tabular}
\caption{Stosunek ilości operacji zmiennoprzecinkowych do ilości operacji odczytu/zapisu z pamięci globalnej -- wersja 2.}
\end{table}

\end{enumerate}
