
\subsection{Wersja 3}

W trzecim podejściu wykorzystana została pamięć współdzielona. W kolejnych iteracjach pętli po blokach najpierw wczytywany jest blok do pamięci współdzielonej (każdy wątek wczytuje jedną komórkę), a następnie wykonywane są obliczenia na dostępnych danych. W tym podejściu niezbędne jest synchronizowanie się wątków dwukrotnie w każdej iteracji.

\lstinputlisting[caption=Mnożenie macierzy kwadratowych na GPU -- wersja 3.]{./code/matrix_multiplication_3.cpp}

\begin{table}[H]
\centering
\begin{tabular}{|c|c|c|c|}
\hline
\multirow{2}{*}{Rozmiar macierzy} & \multicolumn{3}{c|}{Rozmiar bloku} \\ \cline{2-4}
& 8x8 & 16x16 & 22x22 \\ \hline
176x176 & 0.31 & 0.19 & 0.32 \\ \hline
352x352 & 2.33 & 1.27 & 2.23 \\ \hline
528x528 & 7.66 & 4.07 & 7.39 \\ \hline
\end{tabular}
\caption{Czas obliczeń [ms] -- wersja 3.}
\end{table}

\begin{table}[H]
\centering
\begin{tabular}{|c|c|}
\hline
\multirow{2}{*}{Rozmiar macierzy} & \multicolumn{1}{c|}{Rozmiar bloku} \\ \cline{2-2}
& 16x16 \\ \hline
64x64 & 0.02 \\ \hline
128x128 & 0.09 \\ \hline
256x256 & 0.52 \\ \hline
384x384 & 1.62 \\ \hline
512x512 & 3.76 \\ \hline
\end{tabular}
\caption{Czas obliczeń [ms] -- wersja 3.}
\end{table}
